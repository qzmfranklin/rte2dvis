
\documentclass [12pt,letterpaper]{article}
\usepackage{amssymb}
\usepackage{latexsym,bm}
\usepackage{geometry}
\usepackage{graphics}
\usepackage{framed}
\usepackage{subfigure}
\usepackage{multirow}
\usepackage{indentfirst}
\usepackage{amsmath}
\usepackage[english]{babel}
\usepackage{graphicx}
\usepackage{flafter}
\geometry{textwidth=17.5cm,textheight=25cm}


%\usepackage{latexsym,bm}
%\usepackage{graphicx}
%\usepackage{float}
%\usepackage{subfigure}
%\usepackage[boxed]{algorithm2e}
%\usepackage{overpic}
%\usepackage{color}
%\usepackage{mathcal}
%\usepackage[nooneline,flushleft]{caption2}
\usepackage{mathabx}
\usepackage{isomath}
\usepackage{amssymb}
\usepackage{amsmath}
%\usepackage{bm}

%\renewcommand{\textfraction}{0.15}
%\renewcommand{\topfraction}{0.85}
%\renewcommand{\bottomfraction}{0.65}
%\renewcommand{\floatpagefraction}{0.60}
%\newcommand{\subfigwidth}{90mm}
%\newcommand{\subfigheigh}{120mm}
%\newcommand{\subfigwidthw}{90mm}
%\newcommand{\subfigheighw}{65mm}
%\newcommand{\subfigwidthvdx}{120mm}
%\newcommand{\subfigheighvdx}{120mm}
\renewcommand{\vec}{\vectorsym}



\title{2D RTE Integral Solver: Derivation}
\author{Compiled by Zhongming Qu}
\begin{document}
\maketitle

Following the Liemert paper's notation, the radiative transport equation (RTE) reads

\begin{equation}\label{eq:def-RTE}
\hat{\vec{s}}\cdot\nabla\psi(\vec{r},\hat{\vec{s}})+\mu_t(\vec{r})\psi(\vec{r},\hat{\vec{s}})=
\mu_s(\vec{r})\int d\hat{\vec{s}}^\prime f(\hat{\vec{s}}\cdot\hat{\vec{s}}^\prime)
\psi(\vec{r^\prime},\hat{\vec{s}}^\prime)+q(\vec{r},\hat{\vec{s}}),
\end{equation}

with

\begin{center}\begin{tabular}{rll}
      $\vec{r}$&$[L]^{+1}$&\text{position vector}
\\    $\hat{\vec{s}}$&$[L]^{0}$&\text{unit direction vector}
\\    $\psi(\vec{r},\hat{\vec{s}})$&$[L]^{0}$&\text{de-dimensionalized radiance}
\\    $q(\vec{r},\hat{\vec{s}})$&$[L]^{-1}$&\text{source term corresponding to the de-dimensionalized radiance}
\\    $f(\hat{\vec{s}}\cdot\hat{\vec{s}}^\prime)$&$[L]^{0}$&\text{scattering phase function as a function of }$\hat{\vec{s}}\cdot\hat{\vec{s}}^\prime,f_m=g^{\vert m\vert}$
\\    $\mu_s(\vec{r})$&$[L]^{-1}$&\text{scattering cross-section, independent of }$\hat{\vec{s}}$
\\    $\mu_a(\vec{r})$&$[L]^{-1}$&\text{absorption cross-section, independent of }$\hat{\vec{s}}$
\\    $\mu_t=\mu_s+\mu_a$&$[L]^{-1}$&\text{total cross-section, independent of }$\hat{\vec{s}}$
\end{tabular}\end{center}

In case of a single scatterer, $\Omega$, the differential-integral Eq.\eqref{eq:def-RTE} can be transformed
into a volume integral equation (VIE):

\begin{equation}\label{eq:def-VIE}\begin{split}
\psi(\vec{r},\hat{\vec{s}})
&+
\int d\vec{r}^\prime d\hat{\vec{s}}^\prime g(\vec{r},\hat{\vec{s}};\vec{r}^\prime,\hat{\vec{s}}^\prime)
\mu_t(\vec{r}^\prime)\psi(\vec{r}^\prime,\hat{\vec{s}}^\prime)
\\ &-
\int d\vec{r}^\prime d\hat{\vec{s}}^\prime g(\vec{r},\hat{\vec{s}};\vec{r}^\prime,\hat{\vec{s}}^\prime)
\mu_s(\vec{r}^\prime)\int d\hat{\vec{s}}^{\second}
f(\hat{\vec{s}}\cdot\hat{\vec{s}}^{\second})\psi(\vec{r}^\prime,\hat{\vec{s}}^{\second})
\\ &=
\int d\vec{r}^\prime d\hat{\vec{s}}^\prime g(\vec{r},\hat{\vec{s}};\vec{r}^\prime,\hat{\vec{s}}^\prime)
q(\vec{r}^\prime,\hat{\vec{s}}^\prime),
\end{split}\end{equation}

where $g(\vec{r},\hat{\vec{s}};\vec{r}^\prime,\hat{\vec{s}}^\prime)$ ($[L]^{-1}$)
 is the free space ($\mu_a=\mu_s=0$) Green's function:

\begin{equation}\label{eq:2dGF}\begin{split}
g(\vec{r},\hat{\vec{s}};\vec{r}^\prime,\hat{\vec{s}}^\prime)
=
\frac{1}{\vert\vec{r}-\vec{r}^\prime\vert}
\delta(\hat{\vec{s}}-\hat{\vec{s}}_{\vec{r}-\vec{r}^\prime})
\delta(\hat{\vec{s}}^\prime-\hat{\vec{s}}_{\vec{r}-\vec{r}^\prime})
\end{split}\end{equation}

Have used the following notations:

\begin{center}\begin{tabular}{cl}
    $(\vec{r}^\prime,\hat{\vec{s}}^\prime)$&source point position and unit direction vectors \\
    $(\vec{r},\hat{\vec{s}})$&field/observer point position and unit direction vectors \\
    $\hat{\vec{s}}_{\vec{r}-\vec{r}^\prime}$&unit direction vector in the direction of $\vec{r}-\vec{r}^\prime$
\end{tabular}\end{center}

Define the following short-hand notations:

\begin{align}
    \psi&\equiv&\psi(\vec{r},\hat{\vec{s}})& \notag \\
    \psi^S\psi&\equiv&\psi^S[\psi](\vec{r},\hat{\vec{s}})&=
            -\int d\vec{r}^\prime d\hat{\vec{s}}^\prime g(\vec{r},\hat{\vec{s}};\vec{r}^\prime,\hat{\vec{s}}^\prime)
            \mu_t(\vec{r}^\prime)
            \psi(\vec{r}^\prime,\hat{\vec{s}}^\prime) \notag \\
            &&&+\int d\vec{r}^\prime d\hat{\vec{s}}^\prime g(\vec{r},\hat{\vec{s}};\vec{r}^\prime,\hat{\vec{s}}^\prime)
            \mu_s(\vec{r}^\prime)\int d\hat{\vec{s}}^{\second}
            f(\hat{\vec{s}}\cdot\hat{\vec{s}}^{\second})
            \psi(\vec{r}^\prime,\hat{\vec{s}}^{\second}) \label{eq:def-psiS} \\
    \psi^I&\equiv&\psi^I(\vec{r},\hat{\vec{s}})&=
            \int d\vec{r}^\prime d\hat{\vec{s}}^\prime
            g(\vec{r},\hat{\vec{s}};\vec{r}^\prime,\hat{\vec{s}}^\prime)
            q(\vec{r}^\prime,\hat{\vec{s}}^\prime) \label{eq:def-psiI}
\end{align}

Rewrite Eq.\eqref{eq:def-VIE} as $$\psi-\psi^S\psi=\psi^I,$$ which is self-explanatory on its physical meaning: \\ \emph{The total field is the sum of the "incident" field due to the source $q(\vec{r},\hat{\vec{s}})$ in free space and the "scattering field due to the excited sources in the scattering volume}.

By definition, $\psi^S$ is a linear functional acting on $\psi$. $\psi^I(\vec{r}^\prime,\hat{\vec{s}}^\prime)$ is a scalar field determined by the free space Green's function $g(\vec{r},\hat{\vec{s}};\vec{r}^\prime,\hat{\vec{s}}^\prime)$ and the source $q(\vec{r}^\prime,\hat{\vec{s}}^\prime)$. Discretization of Eq.\eqref{eq:def-VIE} will result in a set of linear equations. The solution to the resulting linear equations approximates the radiance field everywhere inside the scattering volume $\Omega$. The radiance field everywhere outside of $\Omega$ can be obtained by Eq.\eqref{eq:def-psiS}

The goal of this paper is the calculate the scattering field due to \emph{plane wave incidence}, i.e., will use $$\psi^I(\vec{r}^\prime,\hat{\vec{s}}^\prime)=\delta(\hat{\vec{s}}-\hat{\vec{s}}^I),$$ where $\hat{\vec{s}}^I$ is the incidence direction unit vector. The singularity of the Dirac delta function leads to significant numerical problems because we will have to be trying to reconstruct some sort of Dirac delta functions. Hereby introduce a technique to tackle this problem.

Write
\begin{equation*}
    \psi(\vec{r},\hat{\vec{s}})=\psi_b(\vec{r},\hat{\vec{s}})+\widetilde{\psi}(\vec{r},\hat{\vec{s}})
\end{equation*}

The subscript $b$ stands for \emph{ballistic}, whose meaning will become clear after this section.

Plug in Eq.\eqref{eq:def-RTE} to get

%
%\begin{equation*}\begin{split}
%    sss
%
%\end{split}\end{equation*}

So far all the derivation applies to both general 2D and 3D problems. 
If we wish to modify something.





























\end{document} 
