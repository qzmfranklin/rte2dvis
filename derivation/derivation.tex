\documentclass [10pt,letterpaper]{article}
\usepackage{hyperref}

\usepackage{amssymb}
\usepackage{amsmath}
\usepackage{latexsym,bm}
\usepackage{geometry}
\usepackage{graphics}
\usepackage{framed}
\usepackage{subfigure}
\usepackage{multirow}
\usepackage{indentfirst}
\usepackage[english]{babel}
\usepackage{graphicx}
\usepackage{flafter}
% page style
\geometry{textwidth=17cm,textheight=22cm}
\usepackage{fancyhdr}
\pagestyle{fancy}
% math
\usepackage{mathabx}
\usepackage{isomath}
\newcommand{\unitvectorsym}[1]{\hat{\vectorsym{#1}}}
\newcommand{\grad}{\nabla}
\renewcommand{\div}{\nabla\cdot}


\begin{document}
\title{2D RTE Volume Integral Solver}
\author{\LaTeX\ by Zhongming Qu}
\maketitle
\tableofcontents
\newpage
\numberwithin{equation}{subsection}

\section{Intro: From RTE to VIE}
\label{sec:intro-from-rte-to-vie}
\subsection{RTE}
\label{sub:rte}
	
This paper follows the notations of Liemert. The radiative transport equation (RTE) reads

\begin{equation}\label{eq:def-RTE}
	\unitvectorsym{s}\cdot\nabla\psi(\vectorsym{r},\unitvectorsym{s})
	+\mu_t(\vectorsym{r})\psi(\vectorsym{r},\unitvectorsym{s})
	=
	\mu_s(\vectorsym{r})
	\int d\unitvectorsym{s}^\prime
	f(\unitvectorsym{s}\cdot\unitvectorsym{s}^\prime)
	\psi(\vectorsym{r^\prime},\unitvectorsym{s}^\prime)
	+q(\vectorsym{r},\unitvectorsym{s})
\end{equation}
with
\begin{center}\begin{tabular}{lll}
      $\vectorsym{r}$&$[L]^{+1}$&\text{position vectorsymtor}
\\    $\unitvectorsym{s}$&$[L]^{0}$&\text{unit direction vectorsymtor}
\\    $\psi(\vectorsym{r},\unitvectorsym{s})$&$[L]^{0}$&\text{de-dimensionalized radiance}
\\    $q(\vectorsym{r},\unitvectorsym{s})$&$[L]^{-1}$&\text{source term corresponding to the de-dimensionalized radiance}
\\    $f(\unitvectorsym{s}\cdot\unitvectorsym{s}^\prime)$&$[L]^{0}$&\text{scattering phase function as a function of }$\unitvectorsym{s}\cdot\unitvectorsym{s}^\prime,f_m=g^{\lvert m\rvert}$
\\    $\mu_s(\vectorsym{r})$&$[L]^{-1}$&\text{scattering cross-section, independent of }$\unitvectorsym{s}$
\\    $\mu_a(\vectorsym{r})$&$[L]^{-1}$&\text{absorption cross-section, independent of }$\unitvectorsym{s}$
\\    $\mu_t=\mu_s+\mu_a$&$[L]^{-1}$&\text{total cross-section, independent of }$\unitvectorsym{s}$
\end{tabular}\end{center}
\subsection{VIE}
\label{subsec:vie}

For a single scatterer, the RTE can transform into the following volume integral equation (VIE)

\begin{equation}\label{eq:def-VIE}\begin{split}
\psi(\vectorsym{r},\unitvectorsym{s})
&+
\int d\vectorsym{r}^\prime d\unitvectorsym{s}^\prime g(\vectorsym{r},\unitvectorsym{s};\vectorsym{r}^\prime,\unitvectorsym{s}^\prime)
\mu_t(\vectorsym{r}^\prime)\psi(\vectorsym{r}^\prime,\unitvectorsym{s}^\prime)
\\ &-
\int d\vectorsym{r}^\prime d\unitvectorsym{s}^\prime g(\vectorsym{r},\unitvectorsym{s};\vectorsym{r}^\prime,\unitvectorsym{s}^\prime)
\mu_s(\vectorsym{r}^\prime)\int d\unitvectorsym{s}^{\second}
f(\unitvectorsym{s}\cdot\unitvectorsym{s}^{\second})\psi(\vectorsym{r}^\prime,\unitvectorsym{s}^{\second})
\\ &=
\int d\vectorsym{r}^\prime d\unitvectorsym{s}^\prime g(\vectorsym{r},\unitvectorsym{s};\vectorsym{r}^\prime,\unitvectorsym{s}^\prime)
q(\vectorsym{r}^\prime,\unitvectorsym{s}^\prime)
\end{split}\end{equation}

where $g(\vectorsym{r},\unitvectorsym{s};\vectorsym{r}^\prime,\unitvectorsym{s}^\prime)$ is the free space, i.e., $\mu_s=\mu_a=0$, Green's function.
Define two linear functionals:

\begin{equation} \label{eq:def-psiI-general}
    \psi^I[q](\vectorsym{r},\unitvectorsym{s})=
    \int d\vectorsym{r}^\prime d\unitvectorsym{s}^\prime
    g(\vectorsym{r},\unitvectorsym{s};\vectorsym{r}^\prime,\unitvectorsym{s}^\prime)
    q(\vectorsym{r}^\prime,\unitvectorsym{s}^\prime)
\end{equation}

\begin{equation} \label{eq:def-psiS-general}
	\begin{split}
	    \psi^S[\psi](\vectorsym{r},\unitvectorsym{s})=
	    &-\int d\vectorsym{r}^\prime d\unitvectorsym{s}^\prime 
	    g(\vectorsym{r},\unitvectorsym{s};\vectorsym{r}^\prime,\unitvectorsym{s}^\prime)
	    \mu_t(\vectorsym{r}^\prime)
	    \psi(\vectorsym{r}^\prime,\unitvectorsym{s}^\prime) \\
	    &+\int d\vectorsym{r}^\prime d\unitvectorsym{s}^\prime 
	    g(\vectorsym{r},\unitvectorsym{s};\vectorsym{r}^\prime,\unitvectorsym{s}^\prime)
	    \mu_s(\vectorsym{r}^\prime)\int d\unitvectorsym{s}^{\second}
	    f(\unitvectorsym{s}\cdot\unitvectorsym{s}^{\second})
	    \psi(\vectorsym{r}^\prime,\unitvectorsym{s}^{\second})
	\end{split}
\end{equation}


Rewrite Eq.\eqref{eq:def-VIE} as

\begin{equation}\label{eq:short-VIE}
	\psi-\psi^S\psi=\psi^I q
\end{equation}

The explicit physical meaning of Eq.\eqref{eq:short-VIE} is: \emph{The total field is the sum of the incident field due to the source $q(\vectorsym{r},\unitvectorsym{s})$ in free space and the scattering field due to the excited sources in the scattering volume}.

By definition, $\psi^S$ is a linear functional acting on $\psi$. $\psi^I(\vectorsym{r}^\prime,\unitvectorsym{s}^\prime)$ is a scalar field determined by the free space Green's function $g(\vectorsym{r},\unitvectorsym{s};\vectorsym{r}^\prime,\unitvectorsym{s}^\prime)$ and the source $q(\vectorsym{r}^\prime,\unitvectorsym{s}^\prime)$.
The free space Green's function is
\begin{equation} \label{eq:exp-gf-general}
	g_t(\vectorsym{r},\unitvectorsym{s};\vectorsym{r}^\prime,\unitvectorsym{s}^\prime)
	=
	\frac
	{1}
	{\lvert \vectorsym{r}-\vectorsym{r}^\prime \rvert}
	\delta(\unitvectorsym{s}-\unitvectorsym{s}_{\vectorsym{r}-\vectorsym{r}^\prime})
	\delta(\unitvectorsym{s}^\prime-\unitvectorsym{s}_{\vectorsym{r}-\vectorsym{r}^\prime})
\end{equation}

Discretization of Eq.\eqref{eq:def-VIE} will result in a set of linear equations. The solution to the resulting linear equations approximates the radiance field everywhere inside the scattering volume
$\Omega$. The radiance field everywhere outside of $\Omega$ can be calculated using Eq.\eqref{eq:def-psiS-general}

The goal of this paper is to calculate the scattering field due to \emph{plane wave incidence} coming in a given direction $\unitvectorsym{s}$, i.e., this paper will use
\begin{equation}\label{eq:exp-psiI-general}
	\psi^I(\vectorsym{r}^\prime,\unitvectorsym{s}^\prime)=\delta(\unitvectorsym{s}-\unitvectorsym{s}^I)
\end{equation}
as the incident radiance field. The use of Eq.\eqref{eq:exp-psiI-general} sidestepped the explicit expression of the incident source $q(\vectorsym{r},\unitvectorsym{s})$, which has an infinitely large strength but is located at infinity to give a finite impact on the volume of interest.

\section{2D and 3D Formalism}
\label{sec:2d-and-3d-formalism}
\subsection{3D Formalism}
The equations in section \ref{sec:intro-from-rte-to-vie} are general. We need to specialize them into 2D and 3D. This paper focuses on 2D.
Make the following substitution to equations in section \ref{sec:intro-from-rte-to-vie}:
\begin{align*}
	\vectorsym{r} & \rightarrow(r,\theta_{\vectorsym{r}},\phi_{\vectorsym{r}}) & d\vectorsym{r} &= r^2 \sin\theta_{\vectorsym{r}} dr d\theta_{\vectorsym{r}}d\phi_{\vectorsym{r}} \\
	\unitvectorsym{s} & \rightarrow(\theta,\phi) & d\unitvectorsym{s} &= \sin\theta d\theta d\phi \\
	\unitvectorsym{s}\cdot\unitvectorsym{s}^\prime & \rightarrow \cos(\theta-\theta^\prime)+\cos(\phi-\phi^\prime)\sin\theta\sin\theta^\prime & & 
\end{align*}

\subsection{2D Formalism}
\label{sub:2d-formalism}
Make the following substitution to equations in section \ref{sec:intro-from-rte-to-vie}:
\begin{align*}
	\vectorsym{r} & \rightarrow(r,\phi_{\vectorsym{r}}) & d\vectorsym{r} &= r dr d\phi_{\vectorsym{r}} \\
	\unitvectorsym{s} & \rightarrow\phi & d\unitvectorsym{s} &= d\phi \\
	\unitvectorsym{s}\cdot\unitvectorsym{s}^\prime & \rightarrow \cos(\phi-\phi^\prime), \phi-\phi^\prime
\end{align*}

The 2D RTE, VIE, and related equations are:
\begin{equation} \label{eq:def-rte-2d}
	\unitvectorsym{s}\cdot\grad\psi(\vectorsym{r},\phi)
	+\mu_t(\vectorsym{r})\psi(\vectorsym{r},\phi)
	=\mu_s(\vectorsym{r})\int d\phi^\prime f(\phi-\phi^\prime)\psi(\vectorsym{r},\phi^\prime)
	+q(\vectorsym{r},\phi)
\end{equation}
\begin{equation} \label{eq:short-vie-2d}
	\psi(\vectorsym{r},\phi)
	-\psi^S[\psi](\vectorsym{r},\phi)
	=\psi^I[q](\vectorsym{r},\phi)
\end{equation}
\begin{equation} \label{eq:def-psiS-2d}
	\begin{split}
		\psi^S[\psi](\vectorsym{r},\phi)=
		&-\int d\vectorsym{r}^\prime d\phi^\prime
		g(\vectorsym{r},\phi;\vectorsym{r}^\prime,\phi^\prime)
		\mu_t(\vectorsym{r}^\prime) \psi(\vectorsym{r}^\prime,\phi^\prime)
		\\ 
		&+\int d\vectorsym{r}^\prime d\phi^\prime
		g(\vectorsym{r},\phi;\vectorsym{r}^\prime,\phi^\prime)
		\mu_s(\vectorsym{r}^\prime)
		\int d\phi^\second
		f(\phi^\prime-\phi^\second) \psi(\vectorsym{r}^\prime,\phi^\second)
	\end{split}
\end{equation}
\begin{equation} \label{eq:def-psiI-2d}
	\psi^I[q](\vectorsym{r},\phi)=
	\int d\vectorsym{r}^\prime d\phi^\prime
	g(\vectorsym{r},\phi;\vectorsym{r}^\prime,\phi^\prime)
	q(\vectorsym{r}^\prime,\phi^\prime)
\end{equation}
\begin{equation} \label{eq:exp-gf-2d}
	g(\vectorsym{r},\phi;\vectorsym{r}^\prime,\phi^\prime)
	=
	\frac{1}{\lvert\vectorsym{r}-\vectorsym{r}^\prime\rvert}
	\delta(\phi-\phi_{\vectorsym{r}-\vectorsym{r}^\prime})
	\delta(\phi^\prime-\phi_{\vectorsym{r}-\vectorsym{r}^\prime})
\end{equation}
\begin{equation} \label{eq:exp-psiI-2d}
	\psi^I(\vectorsym{r},\phi) = \delta(\phi-\phi^I) 
\end{equation}

In Eq.\eqref{eq:exp-psiI-2d}, $\phi^I$ is the incidence angle.
%\\
\subsection{2D Heyney-Greenstein Phase Function}
\label{sub:2d-heyney-greenstein-phase-function} 
This paper uses the 2D Heyney-Greenstein (HG) phase function
\begin{equation} \label{eq:hg-phase-function}
	\begin{split}
		f_g(\phi,\phi^\prime)
		=&
		f_g(\phi-\phi^\prime)
		\\
		=&
		\frac{1}{2\pi}
		\frac{1-g^2}{1+g^2-2g\cos(\phi-\phi^\prime)} 
		\\
		=&
		\frac{1}{2\pi} 
		\sum\limits_{m=-\infty}^{+\infty}
		g^{\lvert m \rvert}
		e^{+i m (\phi-\phi^\prime)} 
	\end{split}
\end{equation}

The discrete Fourier components of the 2D HG phase function are
\begin{equation} \label{eq:hg-phase-function-fm}
	f_m
	=
	(f_g)_m
	=
	\int_0^{2\pi}
	d\phi^\prime
	e^{+i m (\phi^\prime-\phi)}
	f_g(\phi-\phi^\prime)
	=g^{\lvert m \rvert}
\end{equation}

The convolution of two 2D HG phase functions is another 2D HG phase function
\begin{equation} \label{convolution-of-hg-phase-functions}
	\begin{split}
		f_{g_1}*f_{g_2}
		=&
		(f_{g_1}*f_{g_2})(\phi-\phi^\prime)
		\\
		=&
		\int d\phi^\second
		f_{g_1}(\phi-\phi^\second)
		f_{g_2}(\phi^\second-\phi^\prime)
		\\
		=&
		\int d\phi^\second
		%f_{g_1}(\phi-\phi^\second)
		\frac{1}{2\pi} 
		\sum\limits_{m_1=-\infty}^{+\infty}
		g_1^{\lvert m_1 \rvert}
		e^{i m_1 (\phi-\phi^\second)}
		%f_{g_2}(\phi^\second-\phi^\prime)
		\frac{1}{2\pi} 
		\sum\limits_{m_2=-\infty}^{+\infty}
		g_2^{\lvert m_2 \rvert}
		e^{i m_2 (\phi^\second-\phi^\prime)}
		\\
		=&
		\frac{1}{2\pi} 
		%\sum\limits_{m_1,m_2}
		\sum\limits_{m_1=-\infty}^{+\infty}
		%(
		\sum\limits_{m_2=-\infty}^{+\infty}
		\frac{1}{2\pi} 
		\int d\phi^\second
		e^{(m_2-m_1)\phi^\second}
		%)
		g_2^{\lvert m_2 \rvert}
		g_1^{\lvert m_1 \rvert}
		e^{+i m_1 \phi}
		e^{-i m_2 \phi^\prime}
		\\
		=&
		\frac{1}{2\pi} 
		\sum\limits_{m_1=-\infty}^{+\infty}
		\sum\limits_{m_2=-\infty}^{+\infty}
		\delta_{m_1,m_2}
		g_2^{\lvert m_2 \rvert}
		g_1^{\lvert m_1 \rvert}
		e^{+i m_1 \phi}
		e^{-i m_2 \phi^\prime}
		\\
		%\overset{m_1=m_2=m}{=}
		%&
		=&
		\frac{1}{2\pi} 
		\sum\limits_{m=-\infty}^{+\infty}
		(g_1 g_2)^{\lvert m \rvert}
		e^{+i m (\phi-\phi^\prime)}
		\\
		=&
		f_{g_1 g_2}
		(\phi-\phi^\prime)
		%\\
		%=&
		%f_{g_1 g_2}
	\end{split}
\end{equation}
\subsection{Things to Do}
\label{sub:things-to-do}

To solve Eq.\eqref{eq:short-vie-2d}, the basic procedure is to
\begin{enumerate}
	\item discretize the scatterer
	\item apply test procedure
	\item calculate matrix elements
	\item solve the linear equations 
\end{enumerate}

Possible accelerating techniques are
\begin{enumerate}
	\setcounter{enumi} 4
	\item matrix compression using low rank tensors
	\item iterative linear solver using M-V multiplication
\end{enumerate}

All the rest of this paper are devoted to each and all of the procedure and techniques mentioned above.


\section{Discretizing the VIE}
\label{sec:discretizing-the-vie}

\subsection{Finite Element Expansion}
\label{sub:finite-element-expansion}

Grid the volume of interest using $N_s$ triangles, numbered as $n=1,2,3,\ldots,N_S$.
Use the following finite series to approximate the true $\psi(\vectorsym{s},\phi)$:
\begin{equation} \label{eq:def-finite-expansion}
	\psi(\vectorsym{r},\phi)
	=\sum\limits_{n=1}^{N_s}
	\sum\limits_{m=-N_d}^{N_d}
	X_{n m}
	\xi_{n m}(\vectorsym{r},\phi)
	%S_n(\vectorsym{r})
	%e^{i m \phi}
\end{equation}

$\{\xi_{n m}\}$ is a set of chosen basis functions. This paper uses the following one:
\begin{subequations} \begin{align} \label{eq:def-basis-function}
	\xi_{n m}(\vectorsym{r},\phi) 
	&= 
	S_n(\vectorsym{r})e^{i m \phi}
	\\
	S_n(\vectorsym{r}) &=
	\begin{cases}
		1, & \vectorsym{r}\in S_n
		\\
		0, & \vectorsym{r}\notin S_n
	\end{cases}
\end{align} \end{subequations}

Eq.\eqref{eq:def-finite-expansion} is hereafter called \emph{the finite expansion} of $\psi(\vectorsym{r},\phi)$.
The number of the degree of freedom (d.o.f.) of this finite expansion is $N_s(2N_d+1)$.
\\

This paper will use the following notations:
\begin{center} \begin{tabular}{lll}
	$S_n(\vectorsym{r})$	&\ & \text{the pulse function}
	\\
	$S_n$		&\ & \text{the n-th triangle}
	\\
	$S(n)$		&\ & \text{the area of the n-th triangle}
	\\
	$R(n)$		&\ & $\sqrt{S(n)/\pi}$
	\\
	$\sum\nolimits_{n,m}$	 &\ & $\sum\nolimits_{n=1}^{N_s} \sum\nolimits_{m=-N_d}^{N_d}$
	\\
	$\xi_{n m}(\vectorsym{r},\phi)$ &\ & $S_n(\vectorsym{r}) e^{i m \phi}$
	\\
	$N_n$		&\ & $N_s$
	\\
	$N_m$		&\ & $2N_d+1$
	\\
	$N_g$		&\ & $N_n N_m=N_s(2N_d+1)$
\end{tabular} \end{center}


\subsection{Garlekin Test Procedure}
\label{sub:garlekin-test-procedure}

We have, in short-hand notations, the finite expansion by Eq.\eqref{eq:def-finite-expansion}:
\begin{equation} \label{eq:short-fintie-expansion}
	\psi(\vectorsym{r},\phi)=\sum\nolimits_{n,m}X_{n m}\xi_{n m}(\vectorsym{r},\phi)
\end{equation}

Plug the finite expansion Eq.\eqref{eq:def-finite-expansion} into Eq.\eqref{eq:short-vie-2d}
\begin{equation} \label{eq:short-vie-2d-finite-expansion}
	\sum\nolimits_{n^\prime,m^\prime}
	X_{n^\prime m^\prime}
	\{
		\xi_{n^\prime m^\prime}(\vectorsym{r},\phi)
		-\psi^S[\xi_{n^\prime m^\prime}](\vectorsym{r},\phi)
	\}
	=\psi^I[q](\vectorsym{r},\phi)
\end{equation}

To find out such a set of $X_{n m}$ that the finite expansion (Eq.\eqref{eq:def-finite-expansion}) best approximates the true $\psi(\vectorsym{r},\phi)$, multiply $\xi^*_{n m}(\vectorsym{r},\phi)$ on both sides of Eq.\eqref{eq:short-vie-2d-finite-expansion} and integrate over $d\vectorsym{r}d\phi$ over the volume of the scatterer.
\begin{equation} \label{eq:der-vie-2d-finite-expansion}
	\begin{split}
		\sum\nolimits_{n^\prime,m^\prime}
		X_{n^\prime m^\prime}
		\{&
			\int d\vectorsym{r} d\phi
			\xi^*_{n m}(\vectorsym{r},\phi)
			\xi_{n^\prime m^\prime}(\vectorsym{r},\phi)
			\\
		-&
			\int d\vectorsym{r} d\phi
			\xi^*_{n m}(\vectorsym{r},\phi)
			\psi^S[\xi_{n^\prime m^\prime}](\vectorsym{r},\phi)
		\}
		\\
		=&
		\int d\vectorsym{r} d\phi
		\xi^*_{n m}(\vectorsym{r},\phi)
		\psi^I(\vectorsym{r},\phi)
		\end{split}
\end{equation}

Plug Eq.\eqref{eq:def-psiS-2d} in Eq.\eqref{eq:der-vie-2d-finite-expansion}, define the following matrices:
\begin{align} \label{eq:def-Z-A-B-V}
	Z_{(n m)(n^\prime m^\prime)}
	=&
	A_{(n m)(n^\prime m^\prime)}
	+B_{(n m)(n^\prime m^\prime)}
	\\
	A_{(n m)(n^\prime m^\prime)}
	=&
	\int d\vectorsym{r}d\phi
	\xi^*_{n m}(\vectorsym{r},\phi)
	\xi_{n^\prime m^\prime}
	(\vectorsym{r},\phi)
	\\
	B_{(n m)(n^\prime m^\prime)}
	=&
	-\int d\vectorsym{r}d\phi
	\xi^*_{n m}(\vectorsym{r},\phi)
	\psi^S
	[\xi_{n^\prime m^\prime}]
	(\vectorsym{r},\phi)
	\\
	V_{(n m)(n^\prime m^\prime)}
	=&
	\int d\vectorsym{r}d\phi
	\xi^*_{n m}(\vectorsym{r},\phi)
	\psi^I(\vectorsym{r},\phi) 
\end{align}

Let
\begin{subequations} \label{eq:def-B-Bt-Bs}
\begin{align}
	B_{(n m)(n^\prime m^\prime)}
	=& 
	B^t_{(n m)(n^\prime m^\prime)}
	+B^s_{(n m)(n^\prime m^\prime)}
	\\
	\begin{split}
		B^t_{(n m)(n^\prime m^\prime)}
		=&
		\int d\vectorsym{r}d\phi
		\xi^*_{n m}(\vectorsym{r},\phi)
		\\
		&\times
		\int d\vectorsym{r}^\prime d\phi^\prime
		g(\vectorsym{r},\phi;\vectorsym{r}^\prime,\phi^\prime)
		\mu_t(\vectorsym{r}^\prime)
		\xi_{n^\prime m^\prime}
		(\vectorsym{r}^\prime,\phi^\prime)
	\end{split}
	\\
	\begin{split}
		B^s_{(n m)(n^\prime m^\prime)}
		=&
		-\int d\vectorsym{r}d\phi
		\xi^*_{n m}(\vectorsym{r},\phi)
		\\
		&\times
		\int d\vectorsym{r}^\prime d\phi^\prime
		g(\vectorsym{r},\phi;\vectorsym{r}^\prime,\phi^\prime)
		\mu_s(\vectorsym{r}^\prime)
		\\
		&\times
		\int d\phi^\second
		f(\phi^\prime-\phi^\second)
		\xi_{n^\prime m^\prime}
		(\vectorsym{r}^\prime,\phi^\second) 
	\end{split}
\end{align}
\end{subequations}

The VIE Eq.\eqref{eq:short-vie-2d} is converted to a a set of linear equations:
\begin{equation} \label{eq:Z-X-V}
	Z\cdot X=V
\end{equation}

The structure of $Z$ is:
\begin{itemize}
	\item $N_s\times N_s$ block matrices
	\item each block is a $N_d\times N_d$ matrix
\end{itemize}




\section{Calculating Matrix Elements}
\label{sec:calculate-matrix-elements}
\subsection{Identity Term - $A_{(n m)(n^\prime m^\prime)}$ }
\label{sub:identity-term-Anmnpmp}
\begin{equation} \label{eq:der-calculate-Anmnpmp}
	\begin{split}
		A_{(n m)(n^\prime m^\prime)}
		=&
		\int d\vectorsym{r} d\phi
		\xi^*_{n m}
		(\vectorsym{r},\phi)
		\xi_{n^\prime m^\prime}
		(\vectorsym{r},\phi) 
		\\
		=&
		\int d\vectorsym{r}
		S_n(\vectorsym{r})
		S_{n^\prime}(\vectorsym{r})
		\int d\phi
		e^{-i m \phi}
		e^{+i m^\prime \phi} 
	\end{split}
\end{equation}

Use the 1-point quadrature rule to do the integral over $\vectorsym{r}$ and the fact that the integral over $\phi$ is a discrete delta function:
\begin{align*}
	\int d\vectorsym{r}
	S_n(\vectorsym{r})
	S_{n^\prime}(\vectorsym{r})
	=&
	S(n)
	\delta_{n n^\prime}
	\\
	\int d\phi
	e^{-i m \phi}
	e^{+i m^\prime \phi}
	=&
	2\pi
	\delta_{m m^\prime}
\end{align*}

The 1-point result is
\begin{equation} \label{eq:result-Anmnpmp-1pt}
	A_{(n m)(n^\prime m^\prime)}
	=
	2\pi S(n)
	\delta_{n n^\prime}
	\delta_{m m^\prime}
\end{equation}
\subsection{Interaction Term - $B_{(n m)(n^\prime m^\prime)}$}
\label{sub:interaction-term-Bnmnpmp}
\subsubsection{$B^t_{(n m)(n^\prime m^\prime)} \ n\neq n^\prime$}
\label{subsub:Btnmnpmp-off-diagonal}

\begin{equation} \label{eq:result-Btnmnpmp-off-diagonal-1pt}
	\begin{split}
		B^t_{(n m)(n^\prime m^\prime)}
		=&
		\int d\vectorsym{r} d\phi
		S_n(\vectorsym{r})
		e^{-i m \phi}
		\int d\vectorsym{r}^\prime d\phi^\prime
		g(\vectorsym{r},\phi;\vectorsym{r}^\prime,\phi^\prime)
		\mu_t(\vectorsym{r}^\prime)
		S_{n^\prime}(\vectorsym{r}^\prime)
		e^{+i m^\prime \phi^\prime}
		\\
		=&
		\int d\phi
		S(n)
		e^{-i m \phi}
		\int d\phi^\prime
		S(n^\prime)
		g(\vectorsym{r}_n,\phi;\vectorsym{r}_{n^\prime},\phi^\prime)
		\mu_t(n^\prime)
		e^{+i m^\prime \phi^\prime} 
		\\
		=& 
		\mu_t(n^\prime) S(n) S(n^\prime)
		\int d\phi d\phi^\prime
		e^{-i m \phi}
		e^{+i m^\prime \phi^\prime} 
		g(\vectorsym{r}_n,\phi;\vectorsym{r}_{n^\prime},\phi^\prime)
		\\
		=& 
		\mu_t(n^\prime) S(n) S(n^\prime)
		\int d\phi d\phi^\prime
		e^{-i m \phi}
		e^{+i m^\prime \phi^\prime} 
		\frac{1}{\lvert\vectorsym{r}_n-\vectorsym{r}_{n^\prime}\rvert}
		\delta(\phi-\phi_{\vectorsym{r}_n-\vectorsym{r}_{n^\prime}})
		\delta(\phi^\prime-\phi_{\vectorsym{r}_n-\vectorsym{r}_{n^\prime}})
		\\
		=& 
		\mu_t(n^\prime) S(n) S(n^\prime)
		e^{-i m \phi_{\vectorsym{r}_n-\vectorsym{r}_{n^\prime}}}
		e^{+i m^\prime \phi_{\vectorsym{r}_n-\vectorsym{r}_{n^\prime}}} 
		\frac{1}{\lvert\vectorsym{r}_n-\vectorsym{r}_{n^\prime}\rvert}
		\\
		=& 
		\mu_t(n^\prime) 
		\frac{S(n) S(n^\prime)}{\lvert\vectorsym{r}_n-\vectorsym{r}_{n^\prime}\rvert}
		e^{-i (m-m^\prime) \phi_{\vectorsym{r}_n-\vectorsym{r}_{n^\prime}}}
		%e^{+i m^\prime \phi_{\vectorsym{r}_n-\vectorsym{r}_{n^\prime}}} 
		%e^{+i m \phi_{\vectorsym{r}_n-\vectorsym{r}_{n^\prime}}} 
	\end{split}
\end{equation}

\subsubsection{$B^s_{(n m)(n^\prime m^\prime)} \ n\neq n^\prime$}
\label{subsub:Bsnmnpmp-off-diagonal}

\begin{equation} \label{eq:der-Bsnmnpmp-off-diagonal}
	\begin{split}
		B^s_{(n m)(n^\prime m^\prime)}
		=&
		-\int d\vectorsym{r} d\phi
		S_n(\vectorsym{r})
		e^{-i m \phi}
		\int d\vectorsym{r}^\prime d\phi^\prime
		g(\vectorsym{r},\phi;\vectorsym{r}^\prime,\phi^\prime)
		\mu_s(\vectorsym{r}^\prime)
		\int d\phi^\second
		f(\phi^\prime-\phi^\second)
		S_{n^\prime}(\vectorsym{r}^\prime)
		e^{+i m^\prime \phi^\second}
		\\
		=&
		-\int d\vectorsym{r} d\phi
		S_n(\vectorsym{r})
		e^{-i m \phi}
		\int d\vectorsym{r}^\prime d\phi^\prime
		S_{n^\prime}(\vectorsym{r}^\prime)
		e^{+i m^\prime \phi^\prime}
		g(\vectorsym{r},\phi;\vectorsym{r}^\prime,\phi^\prime)
		\mu_s(\vectorsym{r}^\prime)
		\int d\phi^\second
		f(\phi^\prime-\phi^\second)
		e^{+i m^\prime (\phi^\second-\phi^\prime)}
		\\
		=&
		-\int d\vectorsym{r} d\phi
		S_n(\vectorsym{r})
		e^{-i m \phi}
		\int d\vectorsym{r}^\prime d\phi^\prime
		S_{n^\prime}(\vectorsym{r}^\prime)
		e^{+i m^\prime \phi^\prime}
		g(\vectorsym{r},\phi;\vectorsym{r}^\prime,\phi^\prime)
		\mu_s(\vectorsym{r}^\prime)
		g^{\lvert m^\prime \rvert}
	\end{split}
\end{equation}

Compare the last line of Eq.\eqref{eq:der-Bsnmnpmp-off-diagonal} to the first line of Eq.\eqref{eq:result-Btnmnpmp-off-diagonal-1pt}. Make the following replacement in the calculation of $B^t$:
\begin{align*}
	-(minus)	 \rightarrow & +(plus)
	\\
	\mu_t(n^\prime)	 \rightarrow & \mu_s(n^\prime)g^{\lvert m^\prime \rvert}
\end{align*}

The final result for $B^s$ is just
\begin{equation} \label{eq:result-Bsnmnpmp-off-diagonal-1pt}
	B^s_{(n m)(n^\prime m^\prime)}
	=
	-
	\mu_s(n^\prime)g^{\lvert m^\prime \rvert} 
	\frac{S(n) S(n^\prime)}{\lvert\vectorsym{r}_n-\vectorsym{r}_{n^\prime}\rvert}
	e^{-i (m-m^\prime) \phi_{\vectorsym{r}_n-\vectorsym{r}_{n^\prime}}}
\end{equation}

\subsubsection{$B^t_{(n m)(n^\prime m^\prime)} \ n=n^\prime$}
\label{subsub:Btnmnpmp-diagonal}

Still do the 1 point quadrature for the field point $\vectorsym{r}$, but treat the source point $\vectorsym{r}^\prime$ with greater care:
\begin{equation} \label{eq:der-Btnmnpmp-diagonal-1pt-1}
	\begin{split}
		B^t_{(n m)(n^\prime m^\prime)}
		=&
		\int d\vectorsym{r} d\phi
		S_n(\vectorsym{r})
		e^{-i m \phi}
		\int d\vectorsym{r}^\prime d\phi^\prime
		g(\vectorsym{r},\phi;\vectorsym{r}^\prime,\phi^\prime)
		\mu_t(\vectorsym{r}^\prime)
		S_n(\vectorsym{r}^\prime)
		e^{+i m^\prime \phi^\prime}
		\\
		=&
		\mu_t(n) S(n)
		\int d\vectorsym{r}^\prime
		\int d\phi d\phi^\prime
		e^{-i m \phi}
		e^{+i m^\prime \phi^\prime}
		\frac{1}{\lvert\vectorsym{r}_n-\vectorsym{r}^\prime\rvert}
		\delta(\phi-\phi_{\vectorsym{r}_n-\vectorsym{r}^\prime})
		\delta(\phi^\prime-\phi_{\vectorsym{r}_n-\vectorsym{r}^\prime})
		S_n(\vectorsym{r}^\prime)
		\\
		=&
		\mu_t(n) S(n)
		\int d\vectorsym{r}^\prime
		e^{-i (m-m^\prime) \phi_{\vectorsym{r}_n-\vectorsym{r}^\prime}}
		\frac{1}{\lvert\vectorsym{r}_n-\vectorsym{r}^\prime\rvert}
		S_n(\vectorsym{r}^\prime)
	\end{split}
\end{equation}

Change variable
\begin{equation} \label{eq:der-change-variable}
	\begin{split}
		\vectorsym{r}_n-\vectorsym{r}^\prime
		=&
		\vectorsym{R}
		\\
		\lvert \vectorsym{r}_n-\vectorsym{r}^\prime \rvert 
		=&
		\lvert \vectorsym{R} \rvert = R
		\\
		\phi_{\vectorsym{r}_n-\vectorsym{r}^\prime}
		=&
		\phi_{\vectorsym{R}}
		\\
		d\vectorsym{r}^\prime
		=&
		d\vectorsym{R}=R\ dR\ d\phi_{\vectorsym{R}}
	\end{split}
\end{equation}

So that
\begin{equation} \label{eq:der-Btnmnpmp-diagonal-1pt-2}
	\begin{split}
		B^t_{(n m)(n^\prime m^\prime)}
		=& 
		\mu_t(n) S(n)
		\int \phi_{\vectorsym{R}}\ 
		e^{-i (m-m^\prime) \phi_{\vectorsym{r}_n-\vectorsym{r}^\prime}}
		\int R\ dR 
		\frac{1}{R}
		S_n(\vectorsym{r}^\prime)
		\\
		=&
		\mu_t(n) S(n)
		2\pi 
		\delta_{m m^\prime}
		\int dR\  
		S_n(\vectorsym{r}^\prime)
	\end{split}
\end{equation}

Approximate the triangle domain $S_n$ by a circle of radius $R(n)=\sqrt{S(n)/\pi}$
\begin{equation} \label{eq:result-Btnmnpmp-diagonal-1pt}
	B^t_{(n m)(n^\prime m^\prime)}
	=
	2 
	\pi^{1/2}
	S(n)^{3/2}
	\mu_t(n) 
	\delta_{m m^\prime}
\end{equation}

\subsubsection{$B^s_{(n m)(n^\prime m^\prime)} \ n=n^\prime$}
\label{subsub:Bsnmnpmp-diagonal}

Similar to the derivation of off-diagonal $B^s$, the result is easily gotten
\begin{equation} \label{eq:retsult-Bsnmnpmp-diagonal-1p}
	B^s_{(n m)(n^\prime m^\prime)}
	=
	-2 
	\pi^{1/2}
	S(n)^{3/2}
	\mu_s(n)g^{\lvert m\rvert} 
	\delta_{m m^\prime}
\end{equation}

\subsection{Input Vector - $V_{n m}$}
\label{input-vectorsymtor-Vnm}

Use the plane wave incidence field in Eq.\eqref{eq:exp-psiI-2d} to calculate the input vectorsymtor
\begin{equation} \label{eq:result-Vnm-1pt}
	\begin{split}
		V_{n m}
		=&
		\int d\vectorsym{r} d\phi
		\xi^*_{n m} (\vectorsym{r},\phi)
		\psi^I (\vectorsym{r},\phi)
		\\
		=&
		\int d\vectorsym{r} d\phi
		S_n(\vectorsym{r})
		e^{-i m \phi}
		\delta(\phi-\phi^I)
		\\
		\approx&
		S(n)e^{-i m \phi^I}
	\end{split}
\end{equation}

Used 1 point quadrature rule in the last step.
\subsection{$Z=A+B$}
\label{sub:Z-A-B}

For easy reference, list the result of section \ref{sec:calculate-matrix-elements}:
\begin{align} \label{eq:list-of-Z-A-B}
	A_{(n m)(n^\prime m^\prime)}
	=&
	2\pi S(n)
	\delta_{n n^\prime}
	\delta_{m m^\prime}
	\\
	B^t_{(n m)(n^\prime m^\prime)}
	=& 
	\mu_t(n^\prime) 
	\frac{S(n) S(n^\prime)}{\lvert\vectorsym{r}_n-\vectorsym{r}_{n^\prime}\rvert}
	e^{-i (m-m^\prime) \phi_{\vectorsym{r}_n-\vectorsym{r}_{n^\prime}}}
	,
	\forall n\neq n^\prime
	\\
	B^s_{(n m)(n^\prime m^\prime)}
	=&
	-
	\mu_s(n^\prime)g^{\lvert m^\prime \rvert} 
	\frac{S(n) S(n^\prime)}{\lvert\vectorsym{r}_n-\vectorsym{r}_{n^\prime}\rvert}
	e^{-i (m-m^\prime) \phi_{\vectorsym{r}_n-\vectorsym{r}_{n^\prime}}}
	,
	\forall n\neq n^\prime
	\\
	B^t_{(n m)(n^\prime m^\prime)}
	=&
	2 
	\pi^{1/2}
	S(n)^{3/2}
	\mu_t(n) 
	\delta_{m m^\prime}
	,
	\forall n = n^\prime
	\\
	B^s_{(n m)(n^\prime m^\prime)}
	=&
	-2 
	\pi^{1/2}
	S(n)^{3/2}
	\mu_s(n)g^{\lvert m\rvert} 
	\delta_{m m^\prime}
	,
	\forall n = n^\prime
	\\
	V_{n m}
	=&
	S(n)e^{-i m \phi^I}
\end{align}

So that
\begin{equation} \label{eq:result-Znmnpmp-1pt}
	\begin{split}
		Z_{(n m)(n^\prime m^\prime)}
		=&
		A_{(n m)(n^\prime m^\prime)}
		+
		B_{(n m)(n^\prime m^\prime)}
		\\
		=& 
		\begin{cases}
			2\pi S(n) 
			\{
			1+[ \mu_t(n) -\mu_s(n) g^{\lvert m \rvert} ] 
			S(n)^{1/2}\pi^{-1/2}
			%\sqrt{S(n)/\pi}
			\}
			&
			n=n^\prime
			\\
			\frac{S(n)S(n^\prime)}{\lvert \vectorsym{r}_n-\vectorsym{r}_{n^\prime} \rvert}
			[ \mu_t(n^\prime)-\mu_s(n^\prime) g^{\lvert m^\prime \rvert} ]
			e^{-i(m-m^\prime)\phi_{\vectorsym{r}_n-\vectorsym{r}_{n^\prime}}}
			&
			n \neq n^\prime
		\end{cases}
	\end{split}
\end{equation}



\section{Accelerations}
\label{sec:acceleration}
	
\subsection{Acceleration on $N_d$}
\label{sub:acceleration-on-Nd}

By the virtue of Eq.\eqref{eq:result-Znmnpmp-1pt} and Eq.\eqref{eq:result-Vnm-1pt}, in principle the system can be solved by direcetly inverting $Z$. 
But  even for small $N_s$ and $N_d$, the size of matrix $Z$ could become excessively large for computers as of the year 2013.  
This section explains a method for accelerating $N_d$ related calculation. 
The number of floating point operations (flop) is reduced by a factor of $N_m=2N_d+1$. 
That said, we achieved $O(N_d)$ scaling in both CPU and memory.
\\

Treat $Z_{(n m)(n^\prime m^\prime)}$ as $N_s\times N_s$ block matrices of size $N_d\times N_d$. The diagonal $Z$ block matrices are
\begin{equation} \label{eq:def-Znnp-diagonal}
	\begin{split}
		Z(n,n)_{m m^\prime}
		\equiv
		& 
		Z_{(n m)(n^\prime m^\prime)}
		\\
		=&
		\delta_{m m^\prime}
		[
			2\pi S(n)
			+
			2\pi^{1/2}
			\mu_t(n)
			S(n)^{3/2}
			-
			2\pi^{1/2}
			\mu_s(n)
			S(n)^{3/2}
			g^{\lvert m \rvert}
		]
		\\
		=&
		\delta_{m m^\prime}
		[
			Zon1(n)
			+
			Zon2(n)
			g^{\lvert m \rvert}
		]
	\end{split}
\end{equation}

with two 1D vectorsymtors of length $N_s$
\begin{subequations} \label{eq:def-Zon1-Zon2}
	\begin{align}
		Zon1(n)
		=&
		2\pi S(n)
		+
		2\pi^{1/2}
		\mu_t(n)
		S(n)^{3/2}
		\label{eq:def-Zon1}
		\\
		Zon2(n)
		=& 
		-
		2\pi^{1/2}
		\mu_s(n)
		S(n)^{3/2}
		\label{eq:def_Zon2}
	\end{align}
\end{subequations}

The off-diagonal $Z$ block matrices are
\begin{equation} \label{eq:def-Znnp-off-diagonal}
	\begin{split}
		Z(n,n^\prime)_{m m^\prime}
		\equiv &
		Z_{(n m)(n^\prime m^\prime)}
		\\
		=& 
		\frac{S(n) S(n^\prime)}{\lvert \vectorsym{r}_n-\vectorsym{r}_{n^\prime} \rvert} 
		\mu_t(n^\prime)
		e^{-i m \phi_{\vectorsym{r}_n-\vectorsym{r}_{n^\prime}}}
		e^{+i m^\prime \phi_{\vectorsym{r}_n-\vectorsym{r}_{n^\prime}}}
		\\
		&- 
		\frac{S(n) S(n^\prime)}{\lvert \vectorsym{r}_n-\vectorsym{r}_{n^\prime} \rvert} 
		\mu_s(n^\prime)
		e^{-i m \phi_{\vectorsym{r}_n-\vectorsym{r}_{n^\prime}}}
		g^{\lvert m^\prime \rvert} 
		e^{+i m^\prime \phi_{\vectorsym{r}_n-\vectorsym{r}_{n^\prime}}}
		\\
		=& 
		Zoff(n,n^\prime)
		\mu_t(n^\prime)
		[
			v_{n n^\prime}
			\times
			v^*_{n n^\prime} 
		]
		_{m m^\prime}
		\\
		&- 
		Zoff(n,n^\prime)
		\mu_s(n^\prime)
		[
			v_{n n^\prime}
			\times
			(g\ v^*_{n n^\prime})
		]
		_{m m^\prime}
		\\
		&- 
		Zoff(n,n^\prime)
		[
			v_{n n^\prime}
			\times
			u_{n n^\prime}
		]_{m m^\prime}
	\end{split}
\end{equation}

with the following pre-computed quantities
\begin{subequations} \label{eq:def-Zoff-v-u-g}
	\begin{align}
		Zoff(n,n^\prime)
		=&
		\begin{cases}
			\frac
			{S(n)S(n^\prime)}
			{\lvert \vectorsym{r}_n-\vectorsym{r}_{n^\prime} \rvert} 
			&
			n \neq n^\prime
			\\
			0
			&
			n=n^\prime 
		\end{cases}
		\label{eq:def-Zoffnnp}
		\\
		(v_{n n^\prime})_m
		=&
		e^{- m \phi_{\vectorsym{r}_n-\vectorsym{r}_{n^\prime}}}
		\label{eq:def-vnnp}
		\\
		(u_{n n^\prime})_m
		=&
		[
			\mu_t(n^\prime)
			v^*_{n n^\prime}
			-
			\mu_s(n^\prime)
			g
			v^*_{n n^\prime}
		]_m
		\label{eq:def-unnp}
		\\
		(g)_m
		=&
		g^{\lvert m \rvert}
		\\
		(g\ v^*)_m
		=&
		(g)_m
		(v^*_{n n^\prime})_m 
		\label{eq:def-gm}
	\end{align}
\end{subequations}

\subsection{Acceleration on $N_s$ (not yet)}
\label{sub:acceleration-on-Ns}






\subsection{Matrix-Vector Multiplication without Assembling the Matrix}
\label{sub:matrix-vectorsymtor-multiplication-without-assembling-the-matrix}

So far, all discussions focus on \emph{how to calculate the full matrix faster}.
However, when $N_g$ is greater than a few thousands, direct linear solvers fail. 
In that regime, iterative solvers do the job. 
An iterative solver requires not the full matrix, but only the matrix-vectorsymtor (M-V) multiplication rule as the input. 
Factors affecting the performance of an iterative solver include, but not limited to:
\begin{itemize}
	\item the time per call of the M-V multiplication
	\item the condition number of the matrix
	\item the iterative algorithm
	\item possible use of preconditioner
\end{itemize}

The performance, i.e., the time per call, of the M-V multiplication to a very large extent determines the performance of the iterative solver.  
In this subsection, we develop a fast M-V multiplication method for the 2D VIE.
\\

Using Einstein's convention to sum over repeated indices, the input vectorsymtor $X$ and the output vectorsymtor $Y=Z\cdot X$ can be expressed as
\begin{subequations} %\label{eq:def-Xn-Yn}
	\begin{align}
		X(n^\prime)_{m^\prime}
		\equiv
		&
		X_{n^\prime m^\prime}
		\label{eq:def-Xn}
		\\
		Y(n)_{m}
		\equiv
		&
		Y_{n m}
		=
		Z_{(n m)(n^\prime m^\prime)}
		X_{n^\prime m^\prime}
		\label{eq:def-Yn}
	\end{align}
\end{subequations}

Use the $Z(n,n^\prime)$ defined in Eq.\eqref{eq:def-Znnp-diagonal} and Eq.\eqref{eq:def-Znnp-off-diagonal} to rewrite Eq.\eqref{eq:def-Yn} as
\begin{equation} \label{eq:def-Yn-2}
	\begin{split}
		Y(n)_m
		%=&
		%Z(n,n^\prime)_{m m^\prime}
		%X(n^\prime)_{m^\prime}
		%\\
		=& 
		Z(n,n)_{m m^\prime}
		X(n)_{m^\prime}
		+
		\sum\nolimits_{n\neq n^\prime}
		Z(n,n^\prime)_{m m^\prime}
		X(n^\prime)_m^\prime
	\end{split}
\end{equation}

First, do the diagonal term:
\begin{equation} \label{eq:result-Yn-diagonal}
	\begin{split}
		Z(n,n)_{m m^\prime}
		X(n)_{m^\prime}
		=&
		\sum\nolimits_{m^\prime}
		\delta_{m m^\prime}
		[ Zon1(n)+Zon(2)g^{\lvert m \rvert} ]
		X(n)_{m^\prime} 
		\\
		=&
		[ Zon1(n)+Zon(2)g^{\lvert m \rvert} ]
		X(n)_{m} 
		\\
		=&
		Zon1(n)X(n)_m
		+
		Zon2(n)g^{\lvert m \rvert} X(n)_m
		\\
		=&
		[Zon1(n)X(n)+Zon2(n)g\ X(n)]_m
	\end{split}
\end{equation}

Second, do each of the off diagonal terms separately:
\begin{equation} \label{eq:result-Yn-off-diagonal}
	\begin{split}
		Z(n,n^\prime)_{m m^\prime}
		X(n^\prime)_m^\prime
		=&
		\sum\nolimits_{m^\prime}
		Zoff(n,n^\prime)
		(
		v_{n n^\prime}
		\times
		u_{n n^\prime}
		)_{m m^\prime}
		X(n)_{m^\prime}
		\\
		=&
		Zoff(n,n^\prime)
		(v_{n n^\prime})_m
		\sum\nolimits_{m^\prime}
		(
		u_{n n^\prime}
		)_{m m^\prime}
		X(n)_{m^\prime} 
		\\
		=&
		Zoff(n,n^\prime)
		(v_{n n^\prime})_m
		[
			u_{n n^\prime}
			\cdot
			X(n)
		]
	\end{split}
\end{equation}

Plug Eq.\eqref{eq:result-Yn-diagonal} and Eq.\eqref{eq:result-Yn-off-diagonal} into Eq.\eqref{eq:def-Yn-2} while noting Eq.\eqref{eq:def-Zoffnnp} that $Zoff(n,n)=0$, and write the result in the vectorsymtorial form:
\begin{equation} \label{eq:result-Yn}
	\begin{split}
		Y(n)
		&= 
		Zon1(n) X(n)
		+
		Zon2(n) g\ X(n)
		\\
		&+
		\sum \nolimits_{m=-N_d}^{+N_d}
		Zoff(n,n^\prime)
		[ X(n)\cdot u_{n n^\prime} ]
		v_{n n^\prime}
	\end{split}
\end{equation}





\section{Better Convergence}
\label{sec:better-convergence}
\subsection{Convergence Issue}
\label{sub:convergence-issue}

The true, unapproximated plane wave incidence
\begin{equation} \label{eq:harmonic-expansion-of-psiI-2d}
	\psi^I(\vectorsym{r},\phi)
	=
	\delta(\phi-\phi^I)
	=
	\lim_{N_d \to \infty}
	\frac{1}{2\pi} 
	\sum\limits_{m=-N_d}^{N_d}
	e^{i m (\phi-\phi^I)}
\end{equation}

has an infnitely extended discrete Fourier spectrum of equal strength $\frac{1}{2\pi}$.
Thus, the leading order term in the solution also has the Dirac-delta function.
Any truncation at finite $N_d$ would miss out the high frequency components.
Any effort with a finite $N_d$ would run into the trouble of reconstructing a Dirac-delta function, 
which failed us in terms of accuracy (how close were the numerical solutions to the true solution?) and precision (did the numericl solutions convergence to a common value?).
\\

Numerical experiment (awaiting completion) showed some sign of psuedo-convergence in the range $N_d=10\sim20$. 
But the mechanism broke down as $N_d$ went higher - not a sign of convergence.
This seemingly weird phenamenon is understood in detail (not detailed here) as a result of the singularity of the incidence wave.
The way out is to extract whatever singularities in the true solution by hand before feeding the rest, expectedly smoother part into the numerical mechanism.
\\

Section \eqref{sec:better-convergence} notes down the attempt to eliminate the singular terms and achieve better accuracy and precision.
Each following subsections consists of two parts.
The first parts are general formalism.
The second parts are 2D formalism.


\subsection{Extracting $\psi_b$}
\label{sub:extracting-psib}
\subsubsection{General}
\label{subsub:extracting-psib-general}
The RTE reads, in its general form, as in Eq.\eqref{eq:def-RTE}
\begin{equation*}
	\unitvectorsym{s}\cdot\nabla\psi(\vectorsym{r},\unitvectorsym{s})
	+\mu_t(\vectorsym{r})\psi(\vectorsym{r},\unitvectorsym{s})
	=
	\mu_s(\vectorsym{r})
	\int d\unitvectorsym{s}^\prime
	f(\unitvectorsym{s}\cdot\unitvectorsym{s}^\prime)
	\psi(\vectorsym{r^\prime},\unitvectorsym{s}^\prime)
	+q(\vectorsym{r},\unitvectorsym{s})
\end{equation*}
Let
\begin{equation} \label{eq:def-psi-psib-psisc-general}
	\psi(\vectorsym{r},\unitvectorsym{s})
	=
	\psi_b(\vectorsym{r},\unitvectorsym{s})
	+\psi_{sc}(\vectorsym{r},\unitvectorsym{s})
\end{equation}

The subscript \emph{b} stands for \emph{\bf ballistic}.
The subscript \emph{sc} stands for \emph{\bf scattering}.
Plug Eq.\eqref{eq:def-psi-psib-psisc-general} into Eq.\eqref{eq:def-RTE}
\begin{equation} \label{eq:der-extracting-psib-general-1}
	\begin{split}
		\unitvectorsym{s}
		\cdot
		\psi_b(\vectorsym{r},\unitvectorsym{s})
		+
		\mu_t(\vectorsym{r})
		\psi_b(\vectorsym{r},\unitvectorsym{s})
		+&
		\unitvectorsym{s}
		\cdot
		\psi_{sc}(\vectorsym{r},\unitvectorsym{s})
		+
		\mu_t(\vectorsym{r})
		\psi_{sc}(\vectorsym{r},\unitvectorsym{s})
		%=
		\\
		=
		q(\vectorsym{r},\unitvectorsym{s})
		+&
		\mu_s(\vectorsym{r})
		\int d\unitvectorsym{s}^\prime
		f(\unitvectorsym{s}\cdot\unitvectorsym{s}^\prime)
		\psi_{sc}(\vectorsym{r},\unitvectorsym{s}^\prime) 
		+ 
		\mu_s(\vectorsym{r})
		\int d\unitvectorsym{s}^\prime
		f(\unitvectorsym{s}\cdot\unitvectorsym{s}^\prime)
		\psi_{b}(\vectorsym{r},\unitvectorsym{s}^\prime) 
	\end{split}
\end{equation}

Let
\begin{subequations} \label{eq:def-psib-psisc-qb-general}
	\begin{align} 
		\unitvectorsym{s}
		\cdot
		\psi_b(\vectorsym{r},\unitvectorsym{s})
		+
		\mu_t(\vectorsym{r})
		\psi_b(\vectorsym{r},\unitvectorsym{s})
		=&
		q(\vectorsym{r},\unitvectorsym{s})
		\label{eq:def-psib-general}
		\\ 
		%\begin{split}
			%\unitvectorsym{s}
			%\cdot
			%\psi_{sc}(\vectorsym{r},\unitvectorsym{s})
			%+
			%\mu_t(\vectorsym{r})
			%\psi_{sc}(\vectorsym{r},\unitvectorsym{s})
			%=&
			%\mu_s(\vectorsym{r})
			%\int d\unitvectorsym{s}^\prime
			%f(\unitvectorsym{s}\cdot\unitvectorsym{s}^\prime)
			%\psi_{sc}(\vectorsym{r}^\prime,\unitvectorsym{s}^\prime) 
			%%q_b(\vectorsym{r},\unitvectorsym{s}^\prime)
			%\\
			%+& 
			%\mu_s(\vectorsym{r})
			%\int d\unitvectorsym{s}^\prime
			%f(\unitvectorsym{s}\cdot\unitvectorsym{s}^\prime)
			%\psi_{b}(\vectorsym{r}^\prime,\unitvectorsym{s}^\prime) 
		%\end{split}
		\unitvectorsym{s}
		\cdot
		\psi_{sc}(\vectorsym{r},\unitvectorsym{s})
		+
		\mu_t(\vectorsym{r})
		\psi_{sc}(\vectorsym{r},\unitvectorsym{s})
		=&
		\mu_s(\vectorsym{r})
		\int d\unitvectorsym{s}^\prime
		f(\unitvectorsym{s}\cdot\unitvectorsym{s}^\prime)
		\psi_{sc}(\vectorsym{r},\unitvectorsym{s}^\prime) 
		+
		q_b(\vectorsym{r},\unitvectorsym{s})
		\label{eq:def-psisc-general}
		\\
		q_b(\vectorsym{r},\unitvectorsym{s})
		=&
		\mu_s(\vectorsym{r})
		\int d\unitvectorsym{s}^\prime
		f(\unitvectorsym{s}\cdot\unitvectorsym{s}^\prime)
		\psi_{b}(\vectorsym{r},\unitvectorsym{s}^\prime) 
		\label{eq:def-qb-general}
	\end{align}
\end{subequations}

Eq.\eqref{eq:def-psib-psisc-qb-general} is completely equivalent to Eq.\eqref{eq:der-extracting-psib-general-1}.
The solution to Eq.\eqref{eq:def-psib-general}, $\psi_b(\vectorsym{r},\unitvectorsym{s})$, can be calculated using the Green's function for Eq.\eqref{eq:def-psib-general}
\begin{equation} \label{eq:exp-gft-general}
	g_t(\vectorsym{r},\unitvectorsym{s};\vectorsym{r}^\prime,\unitvectorsym{s}^\prime)
	=
	\frac
	{e^{-\tau(\vectorsym{r}^\prime\to\vectorsym{r})}}
	{\lvert \vectorsym{r}-\vectorsym{r}^\prime \rvert}
	\delta(\unitvectorsym{s}-\unitvectorsym{s}_{\vectorsym{r}-\vectorsym{r}^\prime})
	\delta(\unitvectorsym{s}^\prime-\unitvectorsym{s}_{\vectorsym{r}-\vectorsym{r}^\prime})
\end{equation}

The $\tau(\vectorsym{r}^\prime\to\vectorsym{r})$ in Eq.\eqref{eq:exp-gft-general} is the \emph{optical length} from $\vectorsym{r}^\prime$ to $\vectorsym{r}$.
The optical length is defined as the following path integrals:
\begin{subequations} \label{eq:def-optical-length-general}
	\begin{align}
		\tau(\vectorsym{r}^\prime\to\vectorsym{r})
		\equiv
		&
		\int_{
			C_{\vectorsym{r}^\prime\to\vectorsym{r}}
		}
		\mu_t
		dl
		\label{eq:def-optical-length-rp-to-r}
		\\
		\tau(\vectorsym{r},-\unitvectorsym{s}^I)
		\equiv
		&
		\int_{
			C_{\vectorsym{r},-\unitvectorsym{s}^I}
		}
		\mu_t
		dl
		\label{eq:def-optical-length-rp-sI}
	\end{align}
\end{subequations}

In Eq.\eqref{eq:def-optical-length-general},
\begin{center}
	\begin{tabular}{ll}
		$C_{\vectorsym{r}^\prime\to\vectorsym{r}}$
		&
		\text{ is the straight line from $\vectorsym{r}^\prime$ to $\vectorsym{r}$ }
		\\
		$C_{\vectorsym{r},-\unitvectorsym{s}^I}$
		&
		\text{ is the ray that starts from $\vectorsym{r}^\prime$ and in the direction $-\unitvectorsym{s}^I$ }
	\end{tabular}
\end{center}

List Eq.\eqref{eq:exp-gft-general}, Eq.\eqref{eq:exp-gf-general}, Eq.\eqref{eq:def-psiI-general}, and Eq.\eqref{eq:exp-psiI-general} together:
\begin{subequations} \label{eq:der-psib-1}
	\begin{align}
		g(\vectorsym{r},\unitvectorsym{s};\vectorsym{r}^\prime,\unitvectorsym{s}^\prime)
		=&
		\frac
		{1}
		{\lvert \vectorsym{r}-\vectorsym{r}^\prime \rvert}
		\delta(\unitvectorsym{s}-\unitvectorsym{s}_{\vectorsym{r}-\vectorsym{r}^\prime})
		\delta(\unitvectorsym{s}^\prime-\unitvectorsym{s}_{\vectorsym{r}-\vectorsym{r}^\prime})
		\\
		g_t(\vectorsym{r},\unitvectorsym{s};\vectorsym{r}^\prime,\unitvectorsym{s}^\prime)
		=&
		{e^{-\tau(\vectorsym{r}^\prime\to\vectorsym{r})}}
		g(\vectorsym{r},\unitvectorsym{s};\vectorsym{r}^\prime,\unitvectorsym{s}^\prime) 
		\\ 
   		\psi^I[q](\vectorsym{r},\unitvectorsym{s})
		=&
   		\int d\vectorsym{r}^\prime d\unitvectorsym{s}^\prime
   		g(\vectorsym{r},\unitvectorsym{s};\vectorsym{r}^\prime,\unitvectorsym{s}^\prime)
   		q(\vectorsym{r}^\prime,\unitvectorsym{s}^\prime)
		\\
		=&
		\delta
		(\unitvectorsym{s}-\unitvectorsym{s}^I)
	\end{align}
\end{subequations}

The ballistic term $\psi_b$ is
\begin{equation} \label{eq:result-psib-general}
	\begin{split}
		\psi_b(\vectorsym{r},\unitvectorsym{s})
		=&
		\int d\vectorsym{r}^\prime d\unitvectorsym{s}^\prime
		g_t(\vectorsym{r},\unitvectorsym{s};\vectorsym{r}^\prime,\unitvectorsym{s}^\prime) 
		q(\vectorsym{r}^\prime,\unitvectorsym{s}^\prime)
		\\
		=&
		e^{-\tau(\vectorsym{r},-\unitvectorsym{s}^I)}
		\delta(\unitvectorsym{s}-\unitvectorsym{s}^I)
	\end{split}
\end{equation}

\subsubsection{2D}
\label{subsub:extracting-psib-2d}

Write all the equations in section \ref{subsub:extracting-psib-general} in 2D:
\begin{subequations} \label{eq:def-psib-psisc-qb-psiscI-result-psib-exp-gf-gft}
	\begin{align}
		\unitvectorsym{s}
		\cdot 
		\psi_b(\vectorsym{r},\phi)
		+
		\mu_t(\vectorsym{r})
		\psi_b(\vectorsym{r},\phi)
		=&
		q(\vectorsym{r},\phi)
		\label{eq:def-psib-2d}
		\\
		\unitvectorsym{s}
		\cdot 
		\psi_{sc}(\vectorsym{r},\phi)
		+
		\mu_t(\vectorsym{r})
		\psi_{sc}(\vectorsym{r},\phi)
		=&
		\mu_s(\vectorsym{r})
		\int d\phi^\prime
		f(\phi-\phi^\prime)
		\psi_{sc}(\vectorsym{r},\phi^\prime)
		+
		q_b(\vectorsym{r},\phi)
		\label{eq:def-psisc-2d}
		\\
		q_b(\vectorsym{r},\phi)
		=&
		\mu_s(\vectorsym{r})
		\int d\phi^\prime
		f(\phi-\phi^\prime)
		\psi_b(\vectorsym{r},\phi^\prime) 
		\label{eq:def-qb-2d}
		\\
		\begin{split}
			\label{eq:result-psib-2d}
			\psi_b(\vectorsym{r},\phi)
			=&
			\int d\vectorsym{r}^\prime d\phi^\prime
			g_t(\vectorsym{r},\phi;\vectorsym{r}^\prime,\phi^\prime) 
			q(\vectorsym{r}^\prime,\phi^\prime)
			\\
			=&
			e^{-\tau(\vectorsym{r},-\unitvectorsym{s}^I)}
			\delta(\phi-\phi^I) 
		\end{split} 
		\\
		\psi_{sc}^I[q_b](\vectorsym{r},\phi)
		=&
		\int d\vectorsym{r}^\prime d\phi^\prime
		g(\vectorsym{r},\phi;\vectorsym{r}^\prime,\phi^\prime)
		q_b(\vectorsym{r}^\prime,\phi^\prime)
		\label{eq:def-psiscI-2d} 
		\\
		g_t(\vectorsym{r},\phi;\vectorsym{r}^\prime,\phi^\prime)
		=&
		\frac
		{e^{-\tau(\vectorsym{r}^\prime\to\vectorsym{r})}}
		{\lvert \vectorsym{r}-\vectorsym{r}^\prime \rvert}
		\delta(\phi-\phi_{\vectorsym{r}-\vectorsym{r}^\prime})
		\delta(\phi^\prime-\phi_{\vectorsym{r}-\vectorsym{r}^\prime})
		\label{eq:exp-gft-2d} 
		\\
		g(\vectorsym{r},\phi;\vectorsym{r}^\prime,\phi^\prime)
		=&
		\frac
		{1}
		{\lvert \vectorsym{r}-\vectorsym{r}^\prime \rvert}
		\delta(\phi-\phi_{\vectorsym{r}-\vectorsym{r}^\prime})
		\delta(\phi^\prime-\phi_{\vectorsym{r}-\vectorsym{r}^\prime})
		\label{eq:exp-gf-2d} 
	\end{align}
\end{subequations}

Only need to calculate the new input vectorsymtor, $(V_{sc})_{n m}$, to re-use the mechanism developed in the previous sections.
\begin{equation} \label{eq:def-Vscnm-1pt}
	\begin{split}
		(V_{sc})_{n m}
		=&
		\int d\vectorsym{r} d\phi
		\ \xi^*_{n m}
		(\vectorsym{r},\phi)
		\psi_{sc}^I
		[q_b]
		(\vectorsym{r},\phi) 
	\end{split}
\end{equation}

Calculate in the following order
\begin{equation} \label{eq:order-in-which-Vscnm-1pt-is-calculated}
	\psi_b(\vectorsym{r},\phi)
	\to
	q_b(\vectorsym{r},\phi)
	\to
	\psi_{sc}^I(\vectorsym{r},\phi)
	\to
	(V_{sc})_{n m}
\end{equation}
\begin{subequations} \label{eq:result-qb-psiscI-2d-der-Vscnm-1pt}
	\begin{align}
		\begin{split}
			q_b(\vectorsym{r},\phi)
			=&
			\mu_s(\vectorsym{r})
			\int d\phi^\prime
			f(\phi-\phi^\prime)
			\psi_b(\vectorsym{r},\phi^\prime) 
			\\
			=&
			\mu_s(\vectorsym{r})
			\int d\phi^\prime
			f(\phi-\phi^\prime)
			%\psi_b(\vectorsym{r},\phi^\prime) 
			\delta(\phi-\phi^I)
			e^{-\tau(\vectorsym{r},-\unitvectorsym{s}^I)} 
			\\
			=&
			\mu_s(\vectorsym{r})
			f(\phi-\phi^I)
			e^{-\tau(\vectorsym{r},-\unitvectorsym{s}^I)} 
		\end{split}
		\label{eq:result-qb-2d}
		\\
		\begin{split}
			\psi_{sc}^I(\vectorsym{r},\phi)
			=&
			\int d\vectorsym{r}^\prime d\phi^\prime
			g(\vectorsym{r},\phi;\vectorsym{r}^\prime,\phi^\prime)
			q_b(\vectorsym{r}^\prime,\phi^\prime)
			\\
			=&
			\int d\vectorsym{r}^\prime d\phi^\prime
			%g(\vectorsym{r},\phi;\vectorsym{r}^\prime,\phi^\prime)
			\frac
			{1}
			{\lvert \vectorsym{r}-\vectorsym{r}^\prime \rvert}
			\delta(\phi-\phi_{\vectorsym{r}-\vectorsym{r}^\prime})
			\delta(\phi^\prime-\phi_{\vectorsym{r}-\vectorsym{r}^\prime})
			%\\
			%&
			%\times
			%q_b(\vectorsym{r}^\prime,\phi^\prime) 
			\mu_s(\vectorsym{r}^\prime)
			f(\phi^\prime-\phi^I)
			e^{-\tau(\vectorsym{r}^\prime,-\unitvectorsym{s}^I)} 
			\\
			=& 
			\int d\vectorsym{r}^\prime
			\frac
			{1}
			{\lvert \vectorsym{r}-\vectorsym{r}^\prime \rvert}
			\delta(\phi-\phi_{\vectorsym{r}-\vectorsym{r}^\prime})
			%\\
			%&
			%\times
			\mu_s(\vectorsym{r}^\prime)
			f(\phi_{\vectorsym{r}-\vectorsym{r}^\prime}-\phi^I)
			e^{-\tau(\vectorsym{r}^\prime,-\unitvectorsym{s}^I)} 
		\end{split}
		\label{eq:result-psiscI-2d}
		\\
		\begin{split}
			(V_{sc})_{n m}
			=&
			\int d\vectorsym{r} d\phi
			\ \xi^*_{n m}(\vectorsym{r},\phi)
			\psi_{sc}^I(\vectorsym{r},\phi)
			\\
			=&
			\int d\vectorsym{r} d\phi
			%\xi^*_{n m}(\vectorsym{r},\phi)
			\ S_n(\vectorsym{r})
			e^{-i m \phi}
			%\psi_{sc}^I(\vectorsym{r},\phi)
			\int d\vectorsym{r}^\prime
			\frac
			{1}
			{\lvert \vectorsym{r}-\vectorsym{r}^\prime \rvert}
			\delta(\phi-\phi_{\vectorsym{r}-\vectorsym{r}^\prime})
			\mu_s(\vectorsym{r}^\prime)
			f(\phi_{\vectorsym{r}-\vectorsym{r}^\prime}-\phi^I)
			e^{-\tau(\vectorsym{r}^\prime,-\unitvectorsym{s}^I)} 
			\\
			=&
			\int d\vectorsym{r}
			S_n(\vectorsym{r})
			\int d\vectorsym{r}^\prime
			\frac
			{
				e^{-i m \phi_{\vectorsym{r}-\vectorsym{r}^\prime}}
			}
			{\lvert \vectorsym{r}-\vectorsym{r}^\prime \rvert}
			\mu_s(\vectorsym{r}^\prime)
			f(\phi_{\vectorsym{r}-\vectorsym{r}^\prime}-\phi^I)
			e^{-\tau(\vectorsym{r}^\prime,-\unitvectorsym{s}^I)} 
			\\
			=&
			S(n)
			\int d\vectorsym{r}^\prime
			\frac
			{
				e^{-i m \phi_{\vectorsym{r}_n-\vectorsym{r}^\prime}}
			}
			{\lvert \vectorsym{r}_n-\vectorsym{r}^\prime \rvert}
			\mu_s(\vectorsym{r}^\prime)
			f(\phi_{\vectorsym{r}_n-\vectorsym{r}^\prime}-\phi^I)
			e^{-\tau(\vectorsym{r}^\prime,-\unitvectorsym{s}^I)} 
		\end{split}
		\label{eq:der-Vscnm-1pt}
	\end{align}
\end{subequations}
%%We need to do Eq.\eqref{eq:der-Vscnm-1pt} more accurately than the existing shceme could provide.
%Plug Eq.\eqref{eq:hg-phase-function} into Eq.\eqref{eq:der-Vscnm-1pt}
%\begin{equation} \label{eq:der-Vscnm-non-discretized}
	%\begin{split}
		%(V_{sc})_{n m}
		%=&
		%S(n)
		%\int d\vectorsym{r}^\prime
		%\frac
		%{
			%e^{-i m \phi_{\vectorsym{r}_n-\vectorsym{r}^\prime}}
		%}
		%{\lvert \vectorsym{r}_n-\vectorsym{r}^\prime \rvert}
		%\mu_s(\vectorsym{r}^\prime)
		%f(\phi_{\vectorsym{r}_n-\vectorsym{r}^\prime}-\phi^I)
		%e^{-\tau(\vectorsym{r}^\prime,-\unitvectorsym{s}^I)} 
		%\\
		%=& 
		%S(n)
		%\int d\vectorsym{r}^\prime
		%\frac
		%{
			%e^{-i m \phi_{\vectorsym{r}_n-\vectorsym{r}^\prime}}
		%}
		%{\lvert \vectorsym{r}_n-\vectorsym{r}^\prime \rvert}
		%\mu_s(\vectorsym{r}^\prime)
		%%f(\phi_{\vectorsym{r}_n-\vectorsym{r}^\prime}-\phi^I)
		%\frac{1}{2\pi} 
		%\sum\limits_{m^\prime=-\infty}^{+\infty}
		%g^{\lvert m^\prime \rvert}
		%e^{+i m^\prime (\phi_{\vectorsym{r}_n-\vectorsym{r}^\prime}-\phi^I)} 
		%e^{-\tau(\vectorsym{r}^\prime,-\unitvectorsym{s}^I)} 
		%\\
		%=& 
		%S(n)
		%e^{-i m \phi^I}
		%\int R\ dR
		%\frac {1} {R}
		%\ \mu_s(\vectorsym{r}^\prime)
		%e^{-\tau(\vectorsym{r}^\prime,-\unitvectorsym{s}^I)} 
		%\sum\limits_{m^\prime=-\infty}^{+\infty}
		%g^{\lvert m^\prime \rvert}
		%\int_{0}^{2\pi}
		%\frac {d\phi_{\vectorsym{R}}} {2\pi}
		%e^{+i (m^\prime-m) (\phi_{\vectorsym{R}}-\phi^I)} 
		%\\
		%\neq
		%& 
		%S(n)
		%e^{-i m \phi^I}
		%\int dR
		%\ \mu_s(\vectorsym{r}^\prime)
		%e^{-\tau(\vectorsym{r}^\prime,-\unitvectorsym{s}^I)} 
		%\sum\limits_{m^\prime=-\infty}^{+\infty}
		%g^{\lvert m^\prime \rvert}
		%%\int_{0}^{2\pi}
		%%\frac {d\phi_{\vectorsym{R}}} {2\pi}
		%%e^{+i (m^\prime-m) (\phi_{\vectorsym{R}}-\phi^I)} 
		%\delta_{m m^\prime}
		%%\\
		%%=& 
		%%S(n)
		%%g^{\lvert m \rvert}
		%%e^{-i m \phi^I}
		%%\int dR
		%%\ \mu_s(\vectorsym{r}^\prime)
		%%e^{-\tau(\vectorsym{r}^\prime,-\unitvectorsym{s}^I)} 
	%\end{split}
%\end{equation}
%The unequal sign in Eq.\eqref{eq:der-Vscnm-non-discretized} is due to the non-constancy of $\mu_s(\vectorsym{r}^\prime)$.  

The integral over $\vectorsym{r}^\prime$ shall be carried out over the entire volume of interest.
\begin{equation} \label{eq:integral-over-rp-decomposition-into-selfterm-plus-interactionterm}
	\begin{split}
		\int d\vectorsym{r}^\prime
		%=&
		=
		\sum \limits_{n^\prime=1}^{N_s}
		\int d\vectorsym{r}^\prime
		S_{n^\prime}(\vectorsym{r}^\prime)
		%\\
		%=& 
		=
		\int d\vectorsym{r}^\prime
		S_n(\vectorsym{r}^\prime)
		+
		\sum \limits_{n^\prime \neq n}
		\int d\vectorsym{r}^\prime
		S_{n^\prime}(\vectorsym{r}^\prime)
	\end{split}
\end{equation}

The first term in Eq.\eqref{eq:integral-over-rp-decomposition-into-selfterm-plus-interactionterm} is the self-interaction/ diagonal term. Use the same change of variable as in Eq.\eqref{eq:der-change-variable}
\begin{equation} \label{eq:result-diagonal-Vscnm-1pt}
	\begin{split} 
		&
		S(n)
		\int d\vectorsym{r}^\prime
		e^{-i m \phi_{\vectorsym{r}_n-\vectorsym{r}^\prime}}
		\frac
		{1}
		{\lvert \vectorsym{r}_n-\vectorsym{r}^\prime \rvert}
		\mu_s(\vectorsym{r}^\prime)
		f(\phi_{\vectorsym{r}_n-\vectorsym{r}^\prime}-\phi^I)
		e^{-\tau(\vectorsym{r}^\prime,-\unitvectorsym{s}^I)} 
		\\
		=& 
		S(n)
		\int d\vectorsym{R}
		%e^{-i m \phi_{\vectorsym{r}_n-\vectorsym{r}^\prime}}
		\ e^{-i m \phi_{\vectorsym{R}} }
		\frac
		{1}
		%{\lvert \vectorsym{r}_n-\vectorsym{r}^\prime \rvert}
		R
		%\mu_s(\vectorsym{r}^\prime)
		\mu_s(n)
		%f(\phi_{\vectorsym{r}_n-\vectorsym{r}^\prime}-\phi^I)
		f(\phi_{\vectorsym{R}}-\phi^I)
		%e^{-\tau(\vectorsym{r}^\prime,-\unitvectorsym{s}^I)} 
		e^{-\tau(\vectorsym{r}_n,-\unitvectorsym{s}^I)} 
		\\
		=& 
		S(n)
		\mu_s(n)
		e^{-i m \phi^I }
		e^{-\tau(\vectorsym{r}_n,-\unitvectorsym{s}^I)} 
		\int \ dR
		\int \ d\phi_{\vectorsym{R}}
		\ e^{-i m (\phi_{\vectorsym{R}} - \phi^I)}
		f(\phi_{\vectorsym{R}}-\phi^I)
		\\
		=& 
		S(n)
		\mu_s(n)
		g^{\lvert m \rvert}
		e^{-i m \phi^I }
		e^{-\tau(\vectorsym{r}_n,-\unitvectorsym{s}^I)} 
		\sqrt{S(n)/\pi}
		\\
		=& 
		\pi^{-1/2}
		S(n)^{3/2}
		\mu_s(n)
		g^{\lvert m \rvert}
		e^{-i m \phi^I }
		e^{-\tau(\vectorsym{r}_n,-\unitvectorsym{s}^I)} 
	\end{split}
\end{equation}

The second terms in Eq.\eqref{eq:integral-over-rp-decomposition-into-selfterm-plus-interactionterm} are the other-interaction/ off-diagonal terms are each:
\begin{equation} \label{eq:result-off-diagonal-Vscnm-1pt}
	\begin{split} 
		&
		S(n)
		\int d\vectorsym{r}^\prime
		e^{-i m \phi_{\vectorsym{r}_n-\vectorsym{r}^\prime}}
		\frac
		{1}
		{\lvert \vectorsym{r}_n-\vectorsym{r}^\prime \rvert}
		\mu_s(\vectorsym{r}^\prime)
		f(\phi_{\vectorsym{r}_n-\vectorsym{r}^\prime}-\phi^I)
		e^{-\tau(\vectorsym{r}^\prime,-\unitvectorsym{s}^I)} 
		\\
		=&
		\frac
		{S(n)S(n^\prime)}
		{\lvert \vectorsym{r}_n - \vectorsym{r}_{n^\prime} \rvert} 
		\mu_s(n^\prime)
		e^{-i m \phi_{\vectorsym{r}_n - \vectorsym{r}_{n^\prime}}}
		f(\phi_{\vectorsym{r}_n - \vectorsym{r}_{n^\prime}}-\phi^I)
		e^{-\tau(\vectorsym{r}^\prime,-\unitvectorsym{s}^I)}
	\end{split}
\end{equation}

For numerical computation, pre-compute the following two quantities:
\begin{subequations} \label{eq:def-damp-r2rF} 
	\begin{align}
		damp(n)
		=&
		\exp
		{
			[-\tau(\vectorsym{r}_n,-\unitvectorsym{s}^I)]
		}
		\label{eq:def-tau}
		\\
		r2rF(n,n^\prime)
		=&
		f(\phi_{\vectorsym{r}_n-\vectorsym{r}_{n^\prime}} - \phi^I)
		\label{eq:def-r2rR}
	\end{align} 
\end{subequations}

Recall other pre-computed quantities defined in Eq.\eqref{eq:def-Zoff-v-u-g}, the new $(V_{sc})_{n m}$ can be expressed solely by the pre-computed quantities.
\begin{equation} \label{eq:result-Vscnm-1pt}
	\begin{split} 
		(V_{sc})_{n m}
		=&
		\pi^{-1/2}
		S(n)^{3/2}
		\mu_s(n)
		g^{\lvert m \rvert}
		e^{-i m \phi^I }
		e^{-\tau(\vectorsym{r}_n,-\unitvectorsym{s}^I)} 
		\\
		&+
		\sum \nolimits_{n^\prime \neq n}
		\frac
		{S(n)S(n^\prime)}
		{\lvert \vectorsym{r}_n - \vectorsym{r}_{n^\prime} \rvert} 
		\mu_s(n^\prime)
		e^{-i m \phi_{\vectorsym{r}_n - \vectorsym{r}_{n^\prime}}}
		f(\phi_{\vectorsym{r}_n - \vectorsym{r}_{n^\prime}}-\phi^I)
		e^{-\tau(\vectorsym{r}_{n^\prime},-\unitvectorsym{s}^I)} 
		\\
		=&
		\pi^{-1/2}
		S(n)^{1/2}
		\mu_s(n)
		damp(n)
		g_m
		V_{n m}
		\\
		&+
		\sum \nolimits_{n^\prime \neq n}
		Zoff(n,n^\prime)
		r2rF(n,n^\prime)
		damp(n^\prime)
		\mu_s(n^\prime)
		(v_{n n^\prime})_m
		\\
		=&
		\pi^{-1/2}
		S(n)^{1/2}
		\mu_s(n)
		damp(n)
		g_m
		V_{n m}
		\\
		&+
		\sum \nolimits_{n^\prime=1}^{N_s}
		Zoff(n,n^\prime)
		r2rF(n,n^\prime)
		damp(n^\prime)
		\mu_s(n^\prime)
		(v_{n n^\prime})_m
	\end{split}
\end{equation}

The last line in Eq.\eqref{eq:result-Vscnm-1pt} used the fact that $Zoff(n,n)=0$.
\\

Solve
\begin{equation} \label{eq:Z-Xsc-Vsc}
	Z\cdot X_{sc}=V_{sc}
\end{equation}
for $X_{sc}$.
Also, considering the contribution from $\psi_b$,
\begin{equation} \label{eq:result-Xbnm-1pt}
	\begin{split} 
		(X_b)_{n m}
		=&
		\int d\vectorsym{r} d\phi
		\ \xi^*_{n m}(\vectorsym{r},\phi)
		\psi_b(\vectorsym{r},\phi)
		\\
		=& 
		\int d\vectorsym{r} d\phi
		%\ \xi^*_{n m}(\vectorsym{r},\phi)
		\ S_n(\vectorsym{r})
		e^{-i m \phi}
		%\psi_b(\vectorsym{r},\phi)
		e^{-\tau(\vectorsym{r},-\unitvectorsym{s}^I)}
		\delta(\phi-\phi^I) 
		\\
		=& 
		S(n)
		e^{-i m \phi^I}
		e^{-\tau(\vectorsym{r}_n,-\unitvectorsym{s}^I)}
	\end{split}
\end{equation}

The final solution vectorsymtor $X$ is
\begin{equation} \label{eq:X-Xb-Xsc}
	X=X_b+X_{sc}
\end{equation}
%\\
%Some Mathematica codes illustrating matrix and vectorsymtor operations.
%\\
%Input:
%\begin{equation*}
	%\begin{array}{l}
		%\text{ClearAll}[V,G,S,\text{Vnew},v,g,s]
		%\\
		%V=\text{Table}\left[v_{i,j},\{i,2\},\{j,3\}\right];
		%\\
		%G=\text{Table}\left[g_j,\{j,3\}\right];
		%\\
		%S=\text{Table}\left[s_i,\{i,2\}\right];
		%\\
		%\text{Vnew}=S \left(V^{\mathsf{T}} G\right)^{\mathsf{T}};
		%\\ 
		%\text{Vnew // MatrixForm}
	%\end{array}
%\end{equation*}
%Output:
%\begin{equation*}
	%\left(
	%\begin{array}{ccc}
	 %g_1 s_1 v_{1,1} & g_2 s_1 v_{1,2} & g_3 s_1 v_{1,3} \\
	 %g_1 s_2 v_{2,1} & g_2 s_2 v_{2,2} & g_3 s_2 v_{2,3} \\
	%\end{array} 
	%\right)
%\end{equation*}


\subsection{Extracting $\psi_b$ and $\psi_{sb}$}
\label{sub:extracting-psib-and-psisb}
\subsubsection{General}
\label{subsub:extracting-psib-psisb-general}
Numerical experiment showed that the fix in secion \ref{sub:extracting-psib} gets better convergence, but still not good enough.
Besides the leading term $\psi_b$, let's extract the next leading term $\psi_{sb}$ as well.
In this small section, we develop equations in the general case.
In the next small section, we specialize to 2D.
Let
\begin{equation} \label{eq:def-psi-psib-psisb-psit} 
	\psi(\vectorsym{r},\unitvectorsym{s})
	= 
	\psi_b(\vectorsym{r},\unitvectorsym{s})
	+
	\psi_{sb}(\vectorsym{r},\unitvectorsym{s})
	+
	\psi_t(\vectorsym{r},\unitvectorsym{s})
\end{equation}

The subscript \emph{sb} stands for \emph{\bf scattering ballistic}.
The subscript \emph{t} stands for \emph{\bf true}.
Expectation is that $\psi_t$ is smooth enough to be calculated numerically.
\\

Plug Eq.\eqref{eq:def-psi-psib-psisb-psit} into Eq.\eqref{eq:def-RTE} and decompose the resulted equation into the following equivalent set of equations:
\begin{subequations} \label{eq:def-psi-2ndtime-psisb-psit-general-qb-2ndtime-qsb-general}
	\begin{align}
		\unitvectorsym{s} \cdot \grad
		\psi_b(\vectorsym{r},\unitvectorsym{s})
		+&
		\mu_t(\vectorsym{r})
		\psi_b(\vectorsym{r},\unitvectorsym{s})
		=
		q(\vectorsym{r},\unitvectorsym{s}) 
		\label{eq:def-psib-general-2ndtime}
		\\ 
		\unitvectorsym{s} \cdot \grad
		\psi_{sb}(\vectorsym{r},\unitvectorsym{s})
		+&
		\mu_t(\vectorsym{r})
		\psi_{sb}(\vectorsym{r},\unitvectorsym{s})
		=
		q_b(\vectorsym{r},\unitvectorsym{s}) 
		\label{eq:def-psisb-general}
		\\
		\begin{split}
			\unitvectorsym{s} \cdot \grad
			\psi_t(\vectorsym{r},\unitvectorsym{s})
			+&
			\mu_t(\vectorsym{r})
			\psi_t(\vectorsym{r},\unitvectorsym{s})
			\\
			-&
			\mu_s(\vectorsym{r})
			\int d\unitvectorsym{s}^\prime
			f(\unitvectorsym{s} \cdot \unitvectorsym{s}^\prime)
			\psi_t(\vectorsym{r},\unitvectorsym{s}^\prime)
			=
			q_{sb}(\vectorsym{r},\unitvectorsym{s}) 
		\end{split}
		\label{eq:def-psit-general} 
		\\
		q_b(\vectorsym{r},\unitvectorsym{s})
		=&
		\mu_s(\vectorsym{r})
		\int d\unitvectorsym{s}^\prime
		f(\unitvectorsym{s} \cdot \unitvectorsym{s}^\prime)
		\psi_b(\vectorsym{r},\unitvectorsym{s}^\prime)
		\label{eq:def-qb-general-2ndtime}
		\\ 
		q_{sb}(\vectorsym{r},\unitvectorsym{s})
		=&
		\mu_s(\vectorsym{r})
		\int d\unitvectorsym{s}^\prime
		f(\unitvectorsym{s} \cdot \unitvectorsym{s}^\prime)
		\psi_{sb}(\vectorsym{r},\unitvectorsym{s}^\prime)
		\label{eq:def-qsb-general}
	\end{align}
\end{subequations}
$\psi_b(\vectorsym{r},\unitvectorsym{s})$ is given by Eq.\eqref{eq:result-psib-general}:
\begin{equation} \label{eq:result-psib-general-2ndtime}
	\psi_b(\vectorsym{r},\unitvectorsym{s})
	=
	e^{-\tau(\vectorsym{r},-\unitvectorsym{s}^I)}
	\delta(\unitvectorsym{s}-\unitvectorsym{s}^I)
\end{equation}

Similar to the calculation of $\psi(\vectorsym{r},\unitvectorsym{s})$, $\psi_{sb}(\vectorsym{r},\unitvectorsym{s})$ can be calculated using the following equations:
\begin{equation} \label{eq:der-psisb-general}
	\psi_{sb}(\vectorsym{r},\unitvectorsym{s})
	=
	\int d\vectorsym{r}^\prime d\unitvectorsym{s}^\prime
	g_t(\vectorsym{r},\unitvectorsym{s};\vectorsym{r}^\prime,\unitvectorsym{s}^\prime) 
	q_b(\vectorsym{r}^\prime,\unitvectorsym{s}^\prime) 
\end{equation}

The $g_t$ is defined in Eq.\eqref{eq:exp-gft-general}.

\subsubsection{2D}
\label{subsub:extracting-psib-psisb-2d}
Write all the equations in section \ref{subsub:extracting-psib-psisb-general} in 2D:
\begin{subequations} \label{eq:def-psi-2ndtime-psisb-psit-2d-qb-2ndtime-qsb-2d}
	\begin{align}
		\unitvectorsym{s} \cdot \grad
		\psi_b(\vectorsym{r},\phi)
		+&
		\mu_t(\vectorsym{r})
		\psi_b(\vectorsym{r},\phi)
		=
		q(\vectorsym{r},\phi) 
		\label{eq:def-psib-2d-2ndtime}
		\\ 
		\unitvectorsym{s} \cdot \grad
		\psi_{sb}(\vectorsym{r},\phi)
		+&
		\mu_t(\vectorsym{r})
		\psi_{sb}(\vectorsym{r},\phi)
		=
		q_b(\vectorsym{r},\phi) 
		\label{eq:def-psisb-2d}
		\\
		\begin{split}
			\unitvectorsym{s} \cdot \grad
			\psi_t(\vectorsym{r},\phi)
			+&
			\mu_t(\vectorsym{r})
			\psi_t(\vectorsym{r},\phi)
			\\
			-&
			\mu_s(\vectorsym{r})
			\int d\phi^\prime
			f(\phi - \phi^\prime)
			\psi_t(\vectorsym{r},\phi^\prime)
			=
			q_{sb}(\vectorsym{r},\phi) 
		\end{split}
		\label{eq:def-psit-2d} 
		\\
		\begin{split}
			q_b(\vectorsym{r},\phi)
			=&
			\mu_s(\vectorsym{r})
			\int d\phi^\prime
			f(\phi - \phi^\prime)
			\psi_b(\vectorsym{r},\phi^\prime)
			\\
			=&
			\mu_s(\vectorsym{r})
			f(\phi-\phi^I)
			e^{-\tau(\vectorsym{r},-\unitvectorsym{s}^I)} 
		\end{split}
		\label{eq:result-qb-2d-2ndtime}
		\\ 
		q_{sb}(\vectorsym{r},\phi)
		=&
		\mu_s(\vectorsym{r})
		\int d\phi^\prime
		f(\phi - \phi^\prime)
		\psi_{sb}(\vectorsym{r},\phi^\prime)
		\label{eq:def-qsb-2d}
		\\
		\psi_b(\vectorsym{r},\phi)
		=&
		e^{-\tau(\vectorsym{r},-\phi^I)}
		\delta(\phi-\phi^I)
		\label{eq:result-psib-2d-2ndtime} 
		\\
		\psi_{sb}(\vectorsym{r},\phi)
		=&
		\int d\vectorsym{r}^\prime d\phi^\prime
		g_t(\vectorsym{r},\phi;\vectorsym{r}^\prime,\phi^\prime) 
		q_b(\vectorsym{r}^\prime,\phi^\prime) 
		\label{eq:der-psisb-2d} 
		\\
		g_t(\vectorsym{r},\phi;\vectorsym{r}^\prime,\phi^\prime)
		=&
		\frac
		{e^{-\tau(\vectorsym{r}^\prime\to\vectorsym{r})}}
		{\lvert \vectorsym{r}-\vectorsym{r}^\prime \rvert}
		\delta(\phi-\phi_{\vectorsym{r}-\vectorsym{r}^\prime})
		\delta(\phi^\prime-\phi_{\vectorsym{r}-\vectorsym{r}^\prime})
		\label{eq:exp-gft-2d-2ndtime} 
	\end{align}
\end{subequations}

The second line of Eq.\eqref{eq:result-qb-2d-2ndtime} comes from Eq.\eqref{eq:result-qb-2d}.
\\ 

Eq.\eqref{eq:result-psib-2d-2ndtime} is just Eq.\eqref{eq:result-psib-2d}.
\\

Eq.\eqref{eq:exp-gft-2d-2ndtime} is just Eq.\eqref{eq:exp-gft-2d}.
\\

Calculate to the new input vectorsymtor $(V_t)_{n m}$ in the following order:
\begin{equation} \label{eq:order-in-which-Vtnm-1pt-is-calculated}
	\psi_b
	\to
	q_b
	\to
	\psi_{sb}
	\to
	q_{sb}
	\to
	\psi_{sb}^I
	\to
	(V_t)_{n m}
\end{equation}

\begin{subequations} \label{eq:result-psisb-qsb-psisbI-2d}
	\begin{align} 
		\begin{split}
			\psi_{sb}(\vectorsym{r},\phi)
			=&
			\int d\vectorsym{r}^\prime d\phi^\prime
			g_t(\vectorsym{r},\phi;\vectorsym{r}^\prime,\phi^\prime)
			q_b(\vectorsym{r}^\prime,\phi^\prime)
			\\
			=& 
			\int d\vectorsym{r}^\prime d\phi^\prime
			%g_t(\vectorsym{r},\phi;\vectorsym{r}^\prime,\phi^\prime)
			\frac
			{e^{-\tau(\vectorsym{r}^\prime\to\vectorsym{r})}}
			{\lvert \vectorsym{r}-\vectorsym{r}^\prime \rvert}
			\delta(\phi-\phi_{\vectorsym{r}-\vectorsym{r}^\prime})
			\delta(\phi^\prime-\phi_{\vectorsym{r}-\vectorsym{r}^\prime})
			%q_b(\vectorsym{r}^\prime,\phi^\prime)
			\mu_s(\vectorsym{r}^\prime)
			f(\phi^\prime-\phi^I)
			e^{-\tau(\vectorsym{r}^\prime,-\unitvectorsym{s}^I)} 
			\\
			=& 
			\int d\vectorsym{r}^\prime
			\frac
			{e^{-\tau(\vectorsym{r}^\prime\to\vectorsym{r})}}
			{\lvert \vectorsym{r}-\vectorsym{r}^\prime \rvert}
			\delta(\phi-\phi_{\vectorsym{r}-\vectorsym{r}^\prime})
			\mu_s(\vectorsym{r}^\prime)
			f(\phi_{\vectorsym{r}-\vectorsym{r}^\prime}-\phi^I)
			e^{-\tau(\vectorsym{r}^\prime,-\unitvectorsym{s}^I)} 
		\end{split}
		\label{eq:result-psisb-2d}
		\\
		\begin{split} 
			q_{sb}(\vectorsym{r},\phi)
			=&
			\mu_s(\vectorsym{r})
			\int d\phi^\prime
			f(\phi-\phi^\prime)
			\psi_{sb}(\vectorsym{r},\phi^\prime)
			\\
			=&
			\mu_s(\vectorsym{r})
			\int d\phi^\prime
			f(\phi-\phi^\prime)
			\\
			&
			\times
			%\psi_{sb}(\vectorsym{r},\phi^\prime) 
			\int d\vectorsym{r}^\prime
			\frac
			{e^{-\tau(\vectorsym{r}^\prime\to\vectorsym{r})}}
			{\lvert \vectorsym{r}-\vectorsym{r}^\prime \rvert}
			\delta(\phi^\prime-\phi_{\vectorsym{r}-\vectorsym{r}^\prime})
			\mu_s(\vectorsym{r}^\prime)
			f(\phi_{\vectorsym{r}-\vectorsym{r}^\prime}-\phi^I)
			e^{-\tau(\vectorsym{r}^\prime,-\unitvectorsym{s}^I)} 
			\\
			=& 
			\mu_s(\vectorsym{r})
			\int d\vectorsym{r}^\prime
			f(\phi-\phi_{\vectorsym{r}-\vectorsym{r}^\prime})
			\frac
			{e^{-\tau(\vectorsym{r}^\prime\to\vectorsym{r})}}
			{\lvert \vectorsym{r}-\vectorsym{r}^\prime \rvert}
			\mu_s(\vectorsym{r}^\prime)
			f(\phi_{\vectorsym{r}-\vectorsym{r}^\prime}-\phi^I)
			e^{-\tau(\vectorsym{r}^\prime,-\unitvectorsym{s}^I)} 
		\end{split}
		\label{eq:result-qsb-2d}
			\\
		\begin{split}
			\psi_{sb}^I(\vectorsym{r},\phi)
			=&
			\int d\vectorsym{r}^\prime d\phi^\prime
			g(\vectorsym{r},\phi;\vectorsym{r}^\prime,\phi^\prime)
			q_{sb}(\vectorsym{r}^\prime,\phi^\prime)
			\\
			=& 
			\int d\vectorsym{r}^\prime d\phi^\prime
			%g(\vectorsym{r},\phi;\vectorsym{r}^\prime,\phi^\prime)
			\frac
			{e^{-\tau(\vectorsym{r}^\prime\to\vectorsym{r})}}
			{\lvert \vectorsym{r}-\vectorsym{r}^\prime \rvert}
			\delta(\phi-\phi_{\vectorsym{r}-\vectorsym{r}^\prime})
			\delta(\phi^\prime-\phi_{\vectorsym{r}-\vectorsym{r}^\prime})
			\\
			&
			\times
			%q_{sb}(\vectorsym{r}^\prime,\phi^\prime)
			\mu_s(\vectorsym{r}^\prime)
			\int d\vectorsym{r}^\second
			f(\phi^\prime-\phi_{\vectorsym{r}^\prime-\vectorsym{r}^\second})
			\frac
			{e^{-\tau(\vectorsym{r}^\second\to\vectorsym{r}^\prime)}}
			{\lvert \vectorsym{r}^\prime-\vectorsym{r}^\second \rvert}
			\mu_s(\vectorsym{r}^\second)
			f(\phi_{\vectorsym{r}^\prime-\vectorsym{r}^\second}-\phi^I)
			e^{-\tau(\vectorsym{r}^\second,-\unitvectorsym{s}^I)} 
			\\
			=& 
			\int d\vectorsym{r}^\prime d\vectorsym{r}^\second
			\mu_s(\vectorsym{r}^\prime)
			\mu_s(\vectorsym{r}^\second)
			\frac
			{e^{-\tau(\vectorsym{r}^\prime\to\vectorsym{r})}}
			{\lvert \vectorsym{r}-\vectorsym{r}^\prime \rvert}
			\frac
			{e^{-\tau(\vectorsym{r}^\second\to\vectorsym{r}^\prime)}}
			{\lvert \vectorsym{r}^\prime-\vectorsym{r}^\second \rvert}
			e^{-\tau(\vectorsym{r}^\second,-\unitvectorsym{s}^I)} 
			\\
			&
			\times
			f(\phi_{\vectorsym{r}-\vectorsym{r}^\prime}-\phi_{\vectorsym{r}^\prime-\vectorsym{r}^\second})
			f(\phi_{\vectorsym{r}^\prime-\vectorsym{r}^\second}-\phi^I)
			\delta(\phi-\phi_{\vectorsym{r}-\vectorsym{r}^\prime})
		\end{split}
		\label{eq:result-psisbI-2d}
	\end{align}
\end{subequations}

The input vector:
\begin{equation} \label{eq:der-Vtnm-1pt-1}
	\begin{split}
		(V_t)_{n m}
		=&
		\int d\vectorsym{r} d\phi
		\ \xi^*_{n m}(\vectorsym{r},\phi)
		\psi_{sb}^I(\vectorsym{r},\phi)
		\\
		=&
		\int d\vectorsym{r} d\phi
		%\ \xi^*_{n m}(\vectorsym{r},\phi)
		\ S_n(\vectorsym{r})
		e^{-i m \phi}
		%\psi_{sb}^I(\vectorsym{r},\phi)
		\int d\vectorsym{r}^\prime d\vectorsym{r}^\second
		\mu_s(\vectorsym{r}^\prime)
		\mu_s(\vectorsym{r}^\second)
		\frac
		{e^{-\tau(\vectorsym{r}^\prime\to\vectorsym{r})}}
		{\lvert \vectorsym{r}-\vectorsym{r}^\prime \rvert}
		\frac
		{e^{-\tau(\vectorsym{r}^\second\to\vectorsym{r}^\prime)}}
		{\lvert \vectorsym{r}^\prime-\vectorsym{r}^\second \rvert}
		\\
		&
		\times
		e^{-\tau(\vectorsym{r}^\second,-\unitvectorsym{s}^I)} 
		f(\phi_{\vectorsym{r}-\vectorsym{r}^\prime}-\phi_{\vectorsym{r}^\prime-\vectorsym{r}^\second})
		f(\phi_{\vectorsym{r}^\prime-\vectorsym{r}^\second}-\phi^I)
		\delta(\phi-\phi_{\vectorsym{r}-\vectorsym{r}^\prime})
		\\
		=&
		\int d\vectorsym{r}
		\ S_n(\vectorsym{r})
		e^{-i m \phi_{\vectorsym{r}-\vectorsym{r}^\prime}}
		\int d\vectorsym{r}^\prime d\vectorsym{r}^\second
		\mu_s(\vectorsym{r}^\prime)
		\mu_s(\vectorsym{r}^\second)
		\frac
		{e^{-\tau(\vectorsym{r}^\prime\to\vectorsym{r})}}
		{\lvert \vectorsym{r}-\vectorsym{r}^\prime \rvert}
		\frac
		{e^{-\tau(\vectorsym{r}^\second\to\vectorsym{r}^\prime)}}
		{\lvert \vectorsym{r}^\prime-\vectorsym{r}^\second \rvert}
		\\
		&
		\times
		e^{-\tau(\vectorsym{r}^\second,-\unitvectorsym{s}^I)} 
		f(\phi_{\vectorsym{r}-\vectorsym{r}^\prime}-\phi_{\vectorsym{r}^\prime-\vectorsym{r}^\second})
		f(\phi_{\vectorsym{r}^\prime-\vectorsym{r}^\second}-\phi^I)
		\\
		=&
		S(n)
		\int d\vectorsym{r}^\prime d\vectorsym{r}^\second
		e^{-i m \phi_{\vectorsym{r}_n-\vectorsym{r}^\prime}}
		\mu_s(\vectorsym{r}^\prime)
		\mu_s(\vectorsym{r}^\second)
		\frac
		{e^{-\tau(\vectorsym{r}^\prime\to\vectorsym{r}_n)}}
		{\lvert \vectorsym{r}_n-\vectorsym{r}^\prime \rvert}
		\frac
		{e^{-\tau(\vectorsym{r}^\second\to\vectorsym{r}^\prime)}}
		{\lvert \vectorsym{r}^\prime-\vectorsym{r}^\second \rvert}
		\\
		&
		\times
		e^{-\tau(\vectorsym{r}^\second,-\unitvectorsym{s}^I)} 
		f(\phi_{\vectorsym{r}_n-\vectorsym{r}^\prime}-\phi_{\vectorsym{r}^\prime-\vectorsym{r}^\second})
		f(\phi_{\vectorsym{r}^\prime-\vectorsym{r}^\second}-\phi^I)
	\end{split}
\end{equation}

After lengthy calculation,
\begin{equation} \label{eq:result-Vtnm-1pt-sum}
	(V_t)_{n m}
	=
	\sum \limits_{n^\prime, n^\second}
	\text{\{terms to be summed\}}
\end{equation}

The terms to be summed in Eq.\eqref{eq:result-Vtnm-1pt-sum} are:
\begin{subequations} \label{eq:result-Vtnm-1pt-components}
	\begin{align}
		\begin{split} \label{eq:result-Vtnm-1pt-np-n-npp-np}
			n^\prime \neq n
			,&
			n^\second \neq n^\prime
			\\
			&
			S(n)S(n^\prime)S(n^\second)
			\mu_s(n^\prime)
			\mu_s(n^\second) 
			\frac
			{e^{-i m \phi_{\vectorsym{r}_n-\vectorsym{r}_{n^\prime}}}}
			{\lvert \vectorsym{r}_n - \vectorsym{r}_{n^\prime} \rvert}
			\frac
			{e^{-\tau(\vectorsym{r}_{n^\second} \to \vectorsym{r}_{n^\prime})} }
			{\lvert \vectorsym{r}_{n^\prime} - \vectorsym{r}_{n^\second} \rvert}
			\\
			&
			\times
			e^{-\tau(\vectorsym{r}_{n^\second},-\unitvectorsym{s}^I)}
			f(
			\phi_{\vectorsym{r}_n-\vectorsym{r}_{n^\prime}}
			-
			\phi_{\vectorsym{r}_{n^\prime}-\vectorsym{r}_{n^\second}}
			)
			f(
			\phi_{\vectorsym{r}_{n^\prime}-\vectorsym{r}_{n^\second}}
			-
			\phi^I
			)
		\end{split}
		\\
		\begin{split} \label{eq:result-Vtnm-1pt-npn-npp-np}
			n^\prime = n
			,&
			n^\second \neq n^\prime
			\\
			&
			S(n)^{3/2}
			S(n^\second)
			\pi^{-1/2}
			g^{\lvert m \rvert}
			\mu_s(n^\prime)
			\mu_s(n^\second) 
			e^{-i m \phi_{\vectorsym{r}_n-\vectorsym{r}_{n^\prime}}}
			\frac
			{e^{-\tau(\vectorsym{r}_{n^\second} \to \vectorsym{r}_{n^\prime})} }
			{\lvert \vectorsym{r}_{n^\prime} - \vectorsym{r}_{n^\second} \rvert}
			\\
			&
			\times
			e^{-\tau(\vectorsym{r}_{n^\second},-\unitvectorsym{s}^I)}
			f(
			\phi_{\vectorsym{r}_{n^\prime}-\vectorsym{r}_{n^\second}}
			-
			\phi^I
			) 
		\end{split}
		\\
		\begin{split} \label{eq:result-Vtnm-1pt-np-n-nppnp}
			n^\prime \neq n
			,&
			n^\second = n^\prime
			\\
			&
			S(n)
			S(n^\prime)^{3/2}
			\pi^{-1/2}
			\mu_s^2(n^\prime)
			\frac
			{e^{-i m \phi_{\vectorsym{r}_n-\vectorsym{r}_{n^\prime}}}}
			{\lvert \vectorsym{r}_n - \vectorsym{r}_{n^\prime} \rvert}
			\\
			&
			\times
			e^{-\tau(\vectorsym{r}_{n^\prime},-\unitvectorsym{s}^I)}
			f_{g^2}(
			\phi_{\vectorsym{r}_n-\vectorsym{r}_{n^\prime}}
			-
			\phi^I
			)
		\end{split}
		\\
		\begin{split} \label{eq:result-Vtnm-1pt-npn-nppnp}
			n^\prime = n
			,&
			n^\second = n^\prime
			\\
			&
			S(n)^2
			\pi^{-1}
			g^{2 \lvert m \rvert}
			\mu_s^2(n^\prime)
			e^{-i m \phi^I}
			e^{-\tau(\vectorsym{r}_n,-\unitvectorsym{s}^I)}
		\end{split}
	\end{align}
\end{subequations}
\\

Solve
\begin{equation} \label{eq:Z-Xt-Vt}
	Z\cdot X_t = V_t
\end{equation}

for $X_t$.
Also, considering the contributions from $\psi_b$ and $\psi_{sb}$:
\begin{subequations} \label{eq:result-Xbnm-1pt-2ndtime-der-Xsbnm-1pt}
	\begin{align} 
		(X_b)_{n m}
		=&
		S(n)
		e^{-i m \phi^I}
		e^{-\tau(\vectorsym{r}_n,-\unitvectorsym{s}^I)}
		\label{eq:result-Xbnm-1pt-2ndtime}
		\\
		\begin{split}
			(X_{sb})_{n m}
			=&
			\int d\vectorsym{r} d\phi
			\ \xi^*_{n m}(\vectorsym{r},\phi)
			\psi_{sb}(\vectorsym{r},\phi)
			\\
			=& 
			\int d\vectorsym{r} d\phi
			\ S_n(\vectorsym{r})
			e^{-i m \phi}
			\int d\vectorsym{r}^\prime
			\frac
			{e^{-\tau(\vectorsym{r}^\prime\to\vectorsym{r})}}
			{\lvert \vectorsym{r}-\vectorsym{r}^\prime \rvert}
			\\
			&
			\times
			\delta(\phi-\phi_{\vectorsym{r}-\vectorsym{r}^\prime})
			\mu_s(\vectorsym{r}^\prime)
			f(\phi_{\vectorsym{r}-\vectorsym{r}^\prime}-\phi^I)
			e^{-\tau(\vectorsym{r}^\prime,-\unitvectorsym{s}^I)} 
			\\
			=& 
			S(n)
			\int
			d\vectorsym{r}^\prime
			e^{-i m \phi_{\vectorsym{r}-\vectorsym{r}^\prime}}
			\frac
			{e^{-\tau(\vectorsym{r}^\prime\to\vectorsym{r})}}
			{\lvert \vectorsym{r}-\vectorsym{r}^\prime \rvert}
			\\
			&
			\times
			\mu_s(\vectorsym{r}^\prime)
			f(\phi_{\vectorsym{r}-\vectorsym{r}^\prime}-\phi^I)
			e^{-\tau(\vectorsym{r}^\prime,-\unitvectorsym{s}^I)} 
		\end{split}
		\label{eq:der-Xsbnm-1pt}
	\end{align}
\end{subequations}

Eq.\eqref{eq:result-Xbnm-1pt-2ndtime} is just Eq.\eqref{eq:result-Xbnm-1pt}.
\\

The last line of Eq.\eqref{eq:der-Xsbnm-1pt} looks very much the same as Eq.\eqref{eq:der-Vscnm-1pt}, except for the additional factor of $e^{-\tau(\vectorsym{r}^\prime \to \vectorsym{r})}$. 
Go through the similar procedure as from Eq.\eqref{eq:result-diagonal-Vscnm-1pt} to Eq.\eqref{eq:result-Vscnm-1pt} to get the following result:
\begin{equation} \label{eq:result-Xsbnm-1pt}
	\begin{split} 
		(X_{sb})_{n m}
		=&
		\pi^{-1/2}
		S(n)^{3/2}
		\mu_s(n)
		g^{\lvert m \rvert}
		e^{-i m \phi^I }
		e^{-\tau(\vectorsym{r}_n,-\unitvectorsym{s}^I)} 
		\\
		&+
		\sum \nolimits_{n^\prime \neq n}
		\frac
		{S(n)S(n^\prime)}
		{\lvert \vectorsym{r}_n - \vectorsym{r}_{n^\prime} \rvert} 
		\mu_s(n^\prime)
		e^{-i m \phi_{\vectorsym{r}_n - \vectorsym{r}_{n^\prime}}}
		f(\phi_{\vectorsym{r}_n - \vectorsym{r}_{n^\prime}}-\phi^I)
		e^{-\tau(\vectorsym{r}_{n^\prime},-\unitvectorsym{s}^I)} 
		e^{-\tau(\vectorsym{r}_{n^\prime} \to \vectorsym{r}_n)} 
		\\
		=&
		\pi^{-1/2}
		S(n)^{1/2}
		\mu_s(n)
		damp(n)
		g_m
		V_{n m}
		\\
		&+
		\sum \nolimits_{n^\prime=1}^{N_s}
		Zoff(n,n^\prime)
		r2rF(n,n^\prime)
		damp(n^\prime)
		\mu_s(n^\prime)
		(v_{n n^\prime})_m
		r2rdamp(n,n^\prime)
	\end{split}
\end{equation}

In addition to quantities defined in Eq.\eqref{eq:def-Zoff-v-u-g} and Eq.\eqref{eq:def-damp-r2rF}, Eq.\eqref{eq:result-Xsbnm-1pt} defined the following quantity:
\begin{equation}
	r2rdamp(n,n^\prime)
	=
	e^{-\tau(\vectorsym{r}_{n^\prime} \to \vectorsym{r}_n)} 
\end{equation}

The final solution vectorsymtor is
\begin{equation} \label{eq:X-Xb-Xsb-Xt}
	X=X_b+X_{sb}+X_t
\end{equation}

\section{Quadrature Rules}
\label{sec:quadrature-rules}
This section documents the equations good for code implementation.
\subsection{Gaussian Quadrature Rules (not finished yet)}
\label{sub:gaussian-quadrature-rules}
A general 1D integral
\begin{equation} \label{eq:general-1D-integral-dx}
	\int \nolimits_{a}^{b}
	f(x)
	dx
\end{equation}
can be transformed ($x= \frac{a+b}{2} + \frac{b-a}{2} t$) into 
\begin{equation}  \label{eq:general-1D-integral-dt} 
	\frac{b-a}{2}
	\int \nolimits_{-1}^{+1} 
	f(
		\frac{a+b}{2} 
		+
		\frac{b-a}{2}
		t
	)
	dt
\end{equation}
So that we only need to consider the folloing standard integral
\begin{equation} \label{eq:standard-1D-integral-dx}
	\int \nolimits_{-1}^{+1}
	f(x)
	dx 
\end{equation}
\\
Gaussian quadrature rules are well-known.
Gauss-Legenre quadrature rule is used for smooth integrand.
(not finished yet)

\subsection{Singular and Near-Singular Integrations}
\label{subsec:singular-and-near-singular-integrations}
Given an arbitrary triangle with nodes
\begin{equation*}
	\begin{split}
		P_0(x_0,y_0)
		\\
		P_1(x_1,y_1)
		\\
		P_2(x_2,y_2) 
	\end{split}
\end{equation*}
The integrand is singular at $P_0$
\begin{equation} \label{eq:singular-near-singular-integrand-2D}
	\frac{ f(x,y) }{ \sqrt{(x-x_0)^2+(y-y_0)^2} }
\end{equation}
Usually $\sqrt{(x-x_0)^2+(y-y_0)^2}$ is denoted as $R$. 
The above integrand is called $R^{-1}$ singular.
\\
Shift the origin to $P_0$, then rotate the axes (passive perspective of rotation) so that the $y^{\prime}$-axis align with the normal direction of $\overrightarrow{P_1 P_2}$, namely $\unitvectorsym{n}$.
The transformed coordinates are
\begin{equation} \label{eq:singular-near-singular-rotate}
	\begin{pmatrix}
		x^{\prime}					\\
		y^{\prime}					
	\end{pmatrix}
	=
	\frac{1}{\lvert \overrightarrow{P_1 P_2} \rvert}
	\begin{pmatrix} 
		x_2-x_1 	& 	y_2-y_1		\\
		-y_2+y_1 	& 	x_2-x_1		
	\end{pmatrix}
	\begin{pmatrix}
		x-x_0					\\
		y-y_0
	\end{pmatrix}
\end{equation}
$P_1$ and $P_2$ have the following transormed coordinates:
\begin{subequations} \label{eq:result-singular-near-singular-x1p-x2p-y1p-y2p}
	\begin{align}
		x_1^{\prime}	
		=&	
		[
			(x_2-x_1)(x_1-x_0)
			+ 
			(y_2-y_1)(y_1-y_0)
		]
		/{\lvert \overrightarrow{P_1 P_2} \rvert}
		\\ 
		x_2^{\prime}	
		=&	
		[
			(x_2-x_1)(x_2-x_0)
			+ 
			(y_2-y_1)(y_2-y_0)
		]
		/{\lvert \overrightarrow{P_1 P_2} \rvert}
		\\
		y_1^{\prime}
		=y_2^{\prime}
		=& 
		[
			(x_1-x_2)y_0
			+
			(x_2-x_0)y_1
			+
			(x_0-x_1)y_2 
		]
		/{\lvert \overrightarrow{P_1 P_2} \rvert}
		\\
		h
		=&
		y_1^{\prime}
		=
		y_2^{\prime}
	\end{align}
\end{subequations}
Define
Do the following $\sinh^{-1}$ transform
\begin{equation}  \label{eq:def-singular-near-singular-transform-u-v-xp-yp}
	\begin{split}
	       u
	       =&
		       \sinh^{-1}
		       \left(
				\frac
				{x^{\prime}}
				{y^{\prime}}
		       \right)
	       \\
	       v
	       =&
	       		y^{\prime}
	\end{split}
\end{equation}
The inverse transform is:
\begin{equation}  \label{eq:def-singular-near-singular-transform-xp-yp-u-v}
	\begin{split}
		x^{\prime}
		=& 
			v
			\sinh{u}
		\\
		y^{\prime}
		=&
			v
	\end{split}
\end{equation}
$u$ and $v$ are called the transverse and the radial coordinates, respectively.
\\
The Jacobian cancels the $R^{-1}$ divergence in the original integrand:
\begin{equation} \label{eq:result-singular-near-singular-jacobian-u-v-xp-yp}
	J
	=
	\bigg\lvert
		\frac
		{\partial(u,v)}
		{\partial(x^{\prime},y^{\prime})} 
	\bigg\rvert 
	=
	\sqrt{
		{x^{\prime}}^2
		+
		{y^{\prime}}^2
	}
	=
	R
\end{equation}
The triangular area defined by $P_0 P_1 P_2$ in $x-y,x^{\prime}-y^{\prime}$ plane is thus transformed into a rectangular area in $u-v$ plane.
\begin{equation} \label{eq:singular-near-singular-integral-result}
	\begin{split} 
		I
		=&
			\iint 
			\limits_{\Delta P_0 P_1 P_2}
			dx dy
			\frac{ f(x,y) }
			{ \sqrt{(x-x_0)^2+(y-y_0)^2} }
		\\
		=& 
			\iint 
			\limits_{\Delta P_0 P_1 P_2}
			\frac
			{  dx^{\prime} dy^{\prime}  }
			{ \sqrt{{x^{\prime}}^2+{y^{\prime}}^2} }
			f(x(x^{\prime},y^{\prime}),y(x^{\prime},y^{\prime}))
		\\
		=&
			\int 
			\nolimits_{u_<}^{u_>}
			du
			\int 
			\nolimits_{v_<}^{v_>}
			dv
			\ f(x,y)
	\end{split}
\end{equation}
where $u_<(u_>)$ and $v_<(v_>)$ are the smaller(larger) 
one of $u_1,u_2$ and $v_1,v_2$, respectively:
\begin{equation} \label{eq:def-singular-near-singular-u1-u2-v1-v2}
	\begin{split}
		\begin{cases}
			u_1	=	\sinh^{-1}(x_1^{\prime}/h)		\\
			u_2	=	\sinh^{-1}(x_2^{\prime}/h)
		\end{cases}
		\ \ \ \ \ \ \ \ 
		\begin{cases} 
			v_1	=	0					\\
			v_2	=	h
		\end{cases}
	\end{split}
\end{equation}
Use Eq.\eqref{eq:singular-near-singular-rotate} 
and Eq.\eqref{eq:def-singular-near-singular-transform-xp-yp-u-v} 
to express $x,y$ in terms of $u,v$:
\begin{equation} \label{eq:singular-near-singular-x-y-u-v}
	\begin{split} 
		\begin{pmatrix}
			x					\\
			y
		\end{pmatrix}
		=&
		\begin{pmatrix}
			x_0					\\
			y_0
		\end{pmatrix}
		+
		\frac{1}{\lvert \overrightarrow{P_1 P_2} \rvert}
		\begin{pmatrix} 
			x_2-x_1 	& 	-y_2+y_1	\\
			y_2-y_1 	& 	x_2-x_1		
		\end{pmatrix}
		\begin{pmatrix}
			v \sinh{u}				\\
			v
		\end{pmatrix}
	\end{split}
\end{equation}
The standard Gaussian quadrature rules are often given for $(0,1)$ as $\{\xi^i,w^i\}$,
where $\xi^i$ is the $i$-th abscissas an the $w^i$ is the $i$-th weight.
\\
As an example, the $n$-th order Gauss-Legendre rule 
\begin{equation} \label{eq:singular-near-singular-gauss-legendre-rule-example}
	\begin{split} 
		\int \nolimits_{0}^{1}
		f(x)
		dx
		=&
		\sum \limits_{i}^{n}
		w^i
		f(\xi^i)
	\end{split}
\end{equation}
is exact up to order $2n+1$.
\\
For an integral from $a$ to $b$, similar to Eq.\eqref{eq:general-1D-integral-dt}.
\begin{equation} \label{eq:singular-near-singular-general-1D-quadrature}
	\begin{split} 
		\int \nolimits_{a}^{b}
		f(x)
		dx
		=&
			(b-a)
			\int \nolimits_{-1}^{+1} 
			f[a+(b-a)t]
			dt
		\\
		=&
			(b-a)
			\sum \limits_{i}^{n}
			w^i
			f[a+(b-a)\xi^i]
	\end{split}
\end{equation}
\\
For a rectangular area in $u-v$ plane, 
as in Eq.\eqref{eq:singular-near-singular-integral-result},
simply use the product of two independent Gauss-Legendre rules.
Construct two standard Gaussian quadrature rules, i.e., for $(0,1)$, for $u$- and $v$- axis, respectively.
\begin{equation} \label{eq:singular-near-singular-xiui-wui-xivj-wvj}
	\begin{split} 
		\begin{pmatrix}
			u^i					\\
			v^j
		\end{pmatrix}
		=
		\begin{pmatrix}
			u_1
			+ 
			\left( u_2 - u_1 \right)
			\ \xi_u^i				\\
			h
			%v_1
			%+
			%\left( v_2 - v_1 \right) 
			\ \xi_v^j
		\end{pmatrix}
	\end{split}
	\ \ \ \ \ \ \ \ 
	\text{with}
	\ \ \ \ \ \ \ \ 
	\begin{split} 
		\{ \xi_u^i, w_u^i \}_{i=1}^{N_u}
		\\
		\{ \xi_v^j, w_v^j \}_{j=1}^{N_v}
	\end{split}
\end{equation}
Eq.\eqref{eq:singular-near-singular-integral-result} becomes
\begin{equation} \label{eq:singular-near-singular-general-quadrature}
	\begin{split} 
		I
		=& 
			(v_>-v_<)
			(u_>-u_<)
			\sum \nolimits_{i,j}^{N_u,N_v} 
			w_u^i w_v^j
			f(x,y)
			\big\vert
			_{ (i,j) }
		\\
		=&
			\lvert
				h
				\left( u_1 - u_2 \right)
			\rvert
			\sum \nolimits_{i,j}^{N_u,N_v} 
			w_u^i w_v^j
			f(x,y)
			\big\vert
			_{ (i,j) }
	\end{split}
\end{equation}
The $(x,y)$ coordinates can be obtained by:
\begin{subequations} \label{eq:singular-near-singular-x-y-ui-vj}
	\begin{align} 
		\begin{split} 
			\begin{pmatrix} 
				x					\\
				y
			\end{pmatrix}
			\bigg\vert _{ (i,j) }
			=& 
			\begin{pmatrix}
				x_0					\\
				y_0
			\end{pmatrix}
			+
			\frac{1}{\lvert \overrightarrow{P_1 P_2} \rvert}
			\begin{pmatrix} 
				x_2-x_1 	& 	-y_2+y_1	\\
				y_2-y_1 	& 	x_2-x_1		
			\end{pmatrix}
			\begin{pmatrix}
				v^j \sinh{u^i}				\\
				v^j
			\end{pmatrix}
		\end{split}
	\end{align}
\end{subequations}
Note that Eq.\eqref{eq:singular-near-singular-x-y-ui-vj} made no assumption
that $h>0$ or $u_1<u_2$.
This is very suitable for numerical implementation.
\\
Summary of equations for the $\sinh^{-1}$ transform method:
\begin{subequations} \label{eq:singular-near-singular-summary-of-arsh-equations}
	\begin{align}
		\begin{split} 
			I
			=&
				\iint \limits_{\Delta P_0 P_1 P_2}
				dx dy
				\frac{ f(x,y) } { \sqrt{ (x-x_0)^2 + (y-y_0)^2 } }
			\\
			=&
				\lvert
					h
					(u_1-u_2)
				\rvert
				\sum \nolimits_{i,j}^{N_u,N_v} 
				w_u^i w_v^j
				f(x,y)
				\big\vert
				_{ (i,j) }
		\end{split}
		\\
		x_1^{\prime}	
		=&	
			[
				(x_2-x_1)(x_1-x_0)
				+ 
				(y_2-y_1)(y_1-y_0)
			]
			/{\lvert \overrightarrow{P_1 P_2} \rvert}
		\\ 
		x_2^{\prime}	
		=&	
			[
				(x_2-x_1)(x_2-x_0)
				+ 
				(y_2-y_1)(y_2-y_0)
			]
			/{\lvert \overrightarrow{P_1 P_2} \rvert}
		\\
		y_1^{\prime}
		=y_2^{\prime}
		=& 
			[
				(x_1-x_2)y_0
				+
				(x_2-x_0)y_1
				+
				(x_0-x_1)y_2 
			]
			/{\lvert \overrightarrow{P_1 P_2} \rvert}
		\\
		h
		=&
			y_1^{\prime}
		=
			y_2^{\prime}
		\\
		&
		\begin{cases} 
			u_1	=	\sinh^{-1}(x_1^{\prime}/h)		\\
			u_2	=	\sinh^{-1}(x_2^{\prime}/h)
		\end{cases}
		\ \ \ \ 
		\begin{cases} 
			v_1	=	0					\\
			v_2	=	h				
		\end{cases}
		\\ 
		\begin{split} 
			\begin{pmatrix} 
				x					\\
				y
			\end{pmatrix}
			\bigg\vert _{ (i,j) }
			=& 
			\begin{pmatrix}
				x_0					\\
				y_0
			\end{pmatrix}
			+
			\frac{1}{\lvert \overrightarrow{P_1 P_2} \rvert}
			\begin{pmatrix} 
				x_2-x_1 	& 	-y_2+y_1	\\
				y_2-y_1 	& 	x_2-x_1		
			\end{pmatrix}
				%h
			%\begin{pmatrix}
				%\xi_v^j \sinh{u^i}				\\
				%\xi_v^j 
			%\end{pmatrix}
			\begin{pmatrix}
				v^j \sinh{u^i}				\\
				v^j
			\end{pmatrix}
		\end{split}
		\\
		\begin{split} 
			\begin{pmatrix}
				u^i					\\
				v^j
			\end{pmatrix}
			=&
			\begin{pmatrix}
				u_1
				+ 
				( u_2 - u_1 )
				\ \xi_u^i				\\
				h
				\ \xi_v^j
			\end{pmatrix}
		\end{split}
	\end{align}
\end{subequations}
\subsection{Radial-Angular Method}
\label{sub:radial-angular-method}
We may also use the radial-angular method proposed by Khayat and Wilton [Khayat, Wilton, 2008].
The $R$ is general 3D.
The observation point is $(x_0,y_0,z)$ with possible non-vanishing $z$.
\\
Use cylindrical coordinates:
\begin{subequations} \label{eq:def-radial-angular-method-xp-yp-rp-varphip}
	\begin{align} 
		x^{\prime} =& x-x_0 = r^{\prime} \cos{\varphi^{\prime}}	\\
		y^{\prime} =& y-y_0 = r^{\prime} \sin{\varphi^{\prime}}
	\end{align}
\end{subequations}
The radial-angular method used the following transorm for singularity cancellation:
\begin{subequations} \label{eq:def-radial-angular-method-u-v-J}
	\begin{align}
		\begin{split} 
			u 
			=&
				-
				\ln{\tan{\frac{\varphi^{\prime}}{2}}}
			\\
			=&
				\ln\bigg(
					\sqrt{ 1 + \bigg(
						\frac{x^{\prime}}{y^{\prime}} \bigg)^2 }
					+
					\frac{x^{\prime}}{y^{\prime}}
				\bigg) 
			\\
			=&
				\sinh^{-1} \bigg(
					\frac{x^{\prime}}{y^{\prime}}
				\bigg)
		\end{split}
		\\
		v
		=&
			R
			= 
			\sqrt{ {x^{\prime}}^2 + {y^{\prime}}^2 + z^2 }	
		\\
		J
		=&
			\frac{R}{\cosh{u}}
	\end{align}
\end{subequations}
Note that
\begin{equation} \label{eq:radial-angular-method-tanvarphiover2-eu-xp2-yp2-rp2}
	\begin{split} 
		\tan{\frac{\varphi^{\prime}}{2}}	=&	e^{-u}
		\\
		{x^{\prime}}^2 + {y^{\prime}}^2		=&	{r^{\prime}}^2
	\end{split}
\end{equation}
The inverse transform is:
\begin{subequations} \label{eq:radial-angular-method-xp-yp-u-v}
	\begin{align}
		x^{\prime} 
		=& 
			r^{\prime}
			\frac
			{ 1-{\tan{\frac{\varphi^{\prime}}{2}}}^2 }
			{ 1+{\tan{\frac{\varphi^{\prime}}{2}}}^2 }
		=
			\frac{ \sqrt{v^2-z^2} } { \cosh{u} }
			\sinh{u}
		\\
		y^{\prime}
		=& 
			r^{\prime}
			\frac
			{ 2 \tan{\frac{\varphi^{\prime}}{2}} }
			{ 1+{\tan{\frac{\varphi^{\prime}}{2}}}^2 }
		= 
			\frac{ \sqrt{v^2-z^2} } { \cosh{u} }
	\end{align}
\end{subequations}
With $z$, the integral domain in the $u-v$ plane is no longer rectangular.  
\begin{equation} \label{eq:radial-angular-method-uL-uU-vL-uU}
	\begin{split} 
		u_L	=&
			-\ln{\tan{\frac{\varphi^{\prime}_L}{2}}}
			= 
			\ln\bigg(
				\sqrt{1+\bigg(\frac{x_L}{h}\bigg)^2}
				+ \frac{x_L}{h}
			\bigg)
		\\
		u_U	=&
			-\ln{\tan{\frac{\varphi^{\prime}_U}{2}}}
			= 
			\ln\bigg(
				\sqrt{1+\bigg(\frac{x_U}{h}\bigg)^2}
				+ \frac{x_U}{h}
			\bigg)
		\\
		v_L	=&	\lvert z \rvert				\\
		v_U(u)	=&	\sqrt{ z^2 + (h \cosh{u})^2 }		\\ 
	\end{split}
\end{equation}
However, the integral $I$ can still be evaluated using the same set of basic 1D quadrature rules as in Eq.\eqref{eq:singular-near-singular-xiui-wui-xivj-wvj}
\begin{equation} \label{eq:radial-angular-method-quadrature-rule-result}
	\begin{split} 
		I
		=&
		\iint \limits_{\Delta P_0 P_1 P_2}
		dx dy
		\frac{ f(x,y) } { \sqrt{ (x-x_0)^2 + (y-y_0)^2 + z^2 } }
		\\
		=&
		(u_U-u_L)
		\sum \nolimits_{i,j}^{N_u,N_v} 
		\frac{ v_U-v_L }{ \cosh{u^i} }
		w_u^i w_v^j
		f(x,y)
		\big\vert
		_{ (u^i,v^{i,j}) }
		\\
		=&
		(u_U-u_L)
		\sum \nolimits_{i,j}^{N_u,N_v} 
		\bigg [
			\sqrt{ h^2 + \bigg(
				\frac{ \lvert z \rvert } { \cosh{u^i} } \bigg)^2 }
			- \frac{ \lvert z \rvert } { \cosh{u^i} }
		\bigg ]
		w_u^i w_v^j
		f(x,y)
		\big\vert
		_{ (u^i,v^{i,j}) }
	\end{split}
\end{equation}
The resulted summary of equations is:
\begin{subequations} \label{eq:radial-angular-method-summary-of-equations}
	\begin{align}
		\begin{split} 
			I
			=&
			\iint \limits_{\Delta P_0 P_1 P_2}
			dx dy
			\frac{ f(x,y) } { \sqrt{ (x-x_0)^2 + (y-y_0)^2 + z^2 } }
			\\
			=&
			(u_U-u_L)
			\sum \nolimits_{i,j}^{N_u,N_v} 
			W^{i,j}
			f(x,y)
			\big\vert
			_{ (u^i,v^{i,j}) }
		\end{split}
		\\
		W^{i,j}
		=& 
			\big [
				\sqrt{ h^2 + (
					{ \lvert z \rvert } / { \cosh{u^i} } )^2 }
				- { \lvert z \rvert } / { \cosh{u^i} }
			\big ]
			w_u^i w_v^j
		\\
		x_1^{\prime}	
		=&	
		[
			(x_2-x_1)(x_1-x_0)
			+ 
			(y_2-y_1)(y_1-y_0)
		]
		/{\lvert \overrightarrow{P_1 P_2} \rvert}
		\\ 
		x_2^{\prime}	
		=&	
		[
			(x_2-x_1)(x_2-x_0)
			+ 
			(y_2-y_1)(y_2-y_0)
		]
		/{\lvert \overrightarrow{P_1 P_2} \rvert}
		\\
		y_1^{\prime}
		=y_2^{\prime}
		=& 
		[
			(x_1-x_2)y_0
			+
			(x_2-x_0)y_1
			+
			(x_0-x_1)y_2 
		]
		/{\lvert \overrightarrow{P_1 P_2} \rvert}
		\\
		h
		=&
		\lvert y_1^{\prime} \rvert
		=
		\lvert y_2^{\prime} \rvert
		\\
		x_L
		=&
		\min
		\{
			x_1^{\prime},
			x_2^{\prime}
		\}
		\\
		x_U
		=&
		\max
		\{
			x_1^{\prime},
			x_2^{\prime}
		\}
		\\
		u_L	=&
			\ln\big[
				\sqrt{1+({x_L}/{h})^2}
				+ {x_L}/{h}
			\big]
		\\
		u_U	=&
			\ln\big[
				\sqrt{1+({x_U}/{h})^2}
				+ {x_U}/{h}
			\big]
		\\
		v_L	=&	\lvert z \rvert				\\
		v_U(u)	=&	\sqrt{ z^2 + (h \cosh{u})^2 }		\\ 
		\begin{split} 
			\begin{pmatrix} 
				x					\\
				y
			\end{pmatrix}
			\bigg\vert _{ (u^i,v^{i,j}) }
			=& 
			\begin{pmatrix}
				x_0					\\
				y_0
			\end{pmatrix}
			+
			\frac{1}{\lvert \overrightarrow{P_1 P_2} \rvert}
			\begin{pmatrix} 
				x_2-x_1 	& 	-y_2+y_1	\\
				y_2-y_1 	& 	x_2-x_1		
			\end{pmatrix}
			\frac{ \sqrt{{v^{i,j}}^2-z^2} } { \cosh{u^i} } 
			\begin{pmatrix}
				\sinh{u^i}				\\
				1
			\end{pmatrix}
		\end{split}
		\\
		\begin{split} 
			\begin{pmatrix}
				u^i					\\
				v^{i,j}
			\end{pmatrix}
			=&
			=
			\begin{pmatrix} 
				u_L
				+ 
				u_U
				\ \xi_u^i				\\
				v_L
				+ 
				\sqrt{ z^2 + (h \cosh{u^i})^2 }
				\ \xi_v^j				
			\end{pmatrix}
		\end{split}
	\end{align}
\end{subequations}

\subsection{Duffy Transform}
\label{sub:duffy-transform}






\section{Seven Point Quadrature Result}
\label{sec:seven-point-quadrature-result}

\subsection{Quadrature Rules on Triangle Domain}
\label{sub:quadrature-rules-on-triangle-domain}
David Dunavant and Stephen Wandzura are the two important names.
There are well developed C++ codes available:
\\

\url{http://people.sc.fsu.edu/~jburkardt/cpp_src/wandzura/wandzura.html}
\\

\url{http://people.sc.fsu.edu/~jburkardt/cpp_src/dunavant/dunavant.html}
\\

I have made the Dunavant codes compilable by C compilers 
and linked it through LibraryLink into Mathematica.
One can dynamically compute Dunavant rules by calling from Mathematica.
\\

The standard quadrature rules for triangles are expressed as the abscissas and weights for the standard triangle:
\begin{equation} \label{eq:standard-triangle-quadrature-rule}
	\left\{
		\left(
			\xi_i,\zeta_i
		\right)
		,w_i 
	\right\}_{i=1}^{N}
\end{equation}

The standard triangle is the triangle formed by joining the three nodes
\begin{equation} \label{eq:standard-triangle-coordinates}
	\begin{split}
		(0,0)		\\
		(1,0)		\\
		(0,1)
	\end{split}
\end{equation}

The standard weights, $w_i$, are normalized to unity:
\begin{equation} \label{eq:standard-weights-normalized-to-one}
	\sum \limits_{i=1}^{N} w_i = 1
\end{equation}

The integration on general triangle $P_0(x_0,y_0),P_1(x_1,y_1),P_2(x_2,y_2)$ is approximated using the quadrature rule as:
\begin{equation} \label{eq:general-triangle-numerical-integration-quadrature}
	\int dxdy f(x,y)
	=
		\text{Area}
		\sum \limits_{i=1}^{N}
		w_i
		f(x_i,y_i)
\end{equation}

The abscissas are given by
\begin{equation}
	\begin{pmatrix}
		x_i	\\
		y_i
	\end{pmatrix}
	=
		\begin{pmatrix}
			x_1-x_0		&	x_2 - x_0	\\
			y_1-y_0		&	y_2 - y_0
		\end{pmatrix}
		\begin{pmatrix}
			\xi_i	\\
			\zeta_i
		\end{pmatrix}
		+
		\begin{pmatrix}
			x_0	\\
			y_0
		\end{pmatrix}
\end{equation}


\subsection{Calculating Matrix Elements - Source 7 pt Quadrature}
\label{sub:calculating-matrix-elements-source-7pt-quadrature}


\begin{equation} \label{eq:result-Btnmnpmp-off-diagonal-source-7pt}
	\begin{split}
		B^t_{(n m)(n^\prime m^\prime)}
		=&
		\int d\vectorsym{r} d\phi
		S_n(\vectorsym{r})
		e^{-i m \phi}
		\int d\vectorsym{r}^\prime d\phi^\prime
		g(\vectorsym{r},\phi;\vectorsym{r}^\prime,\phi^\prime)
		\mu_t(\vectorsym{r}^\prime)
		S_{n^\prime}(\vectorsym{r}^\prime)
		e^{+i m^\prime \phi^\prime}
		\\
		=&
		\int d\phi
		S(n)
		e^{-i m \phi}
		\int d\phi^\prime
		S(n^\prime)
		g(\vectorsym{r}_n,\phi;\vectorsym{r}_{n^\prime},\phi^\prime)
		\mu_t(n^\prime)
		e^{+i m^\prime \phi^\prime} 
		\\
		=& 
		\mu_t(n^\prime) S(n) S(n^\prime)
		\int d\phi d\phi^\prime
		e^{-i m \phi}
		e^{+i m^\prime \phi^\prime} 
		g(\vectorsym{r}_n,\phi;\vectorsym{r}_{n^\prime},\phi^\prime)
		\\
		=& 
		\mu_t(n^\prime) S(n) S(n^\prime)
		\int d\phi d\phi^\prime
		e^{-i m \phi}
		e^{+i m^\prime \phi^\prime} 
		\frac{1}{\lvert\vectorsym{r}_n-\vectorsym{r}_{n^\prime}\rvert}
		\delta(\phi-\phi_{\vectorsym{r}_n-\vectorsym{r}_{n^\prime}})
		\delta(\phi^\prime-\phi_{\vectorsym{r}_n-\vectorsym{r}_{n^\prime}})
		\\
		=& 
		\mu_t(n^\prime) S(n) S(n^\prime)
		e^{-i m \phi_{\vectorsym{r}_n-\vectorsym{r}_{n^\prime}}}
		e^{+i m^\prime \phi_{\vectorsym{r}_n-\vectorsym{r}_{n^\prime}}} 
		\frac{1}{\lvert\vectorsym{r}_n-\vectorsym{r}_{n^\prime}\rvert}
		\\
		=& 
		\mu_t(n^\prime) 
		\frac{S(n) S(n^\prime)}{\lvert\vectorsym{r}_n-\vectorsym{r}_{n^\prime}\rvert}
		e^{-i (m-m^\prime) \phi_{\vectorsym{r}_n-\vectorsym{r}_{n^\prime}}}
		%e^{+i m^\prime \phi_{\vectorsym{r}_n-\vectorsym{r}_{n^\prime}}} 
		%e^{+i m \phi_{\vectorsym{r}_n-\vectorsym{r}_{n^\prime}}} 
	\end{split}
\end{equation}




\section{Code Implementation}
\label{sec:code-implementation}
%\subsection{Summary of Formulae}
%\label{sub:summary-of-formulae}
\subsection{Mesh Generation}
\label{sub:mesh-generation}
%Used Gmsh to generate mesh file in .msh format.
%Wrote Mathematica functions to dump .msh files into .dat files containing nodes and triangles.
\subsection{Pre-Computation}
\label{sub:pre-computation}
\subsection{Fast M-V Multiplication}
\label{sub:fast-M-V-multiplication}
\subsection{Iterative Solver}
\label{sub:iterative-solver}
\subsection{First Order - $\psi_b$}
\label{sub:first-order-psib}
\subsection{Second Order - $\psi_b$ and $\psi_{sb}$}
\label{sub:second-order-psib-and-psisb} 
\subsection{Quadrature Rules}
\label{sub:quadrature-rules}
\section{What's Next?}
\label{sec:whats-next}




\end{document}
