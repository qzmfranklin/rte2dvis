\documentclass[main]{subfiles} 
\begin{document}
\section{Section2}
\label{sec:Section2}
%%%%%%%%%%%%%%%%%%%%%%%%%%%%%%%%%%%%%%%%%%%%%%%%%%%%%%%%%%%%%%%%%%%%%%%%%%%%%%%%
Expand the specific intensity $\psi(\v{r},\phi)$:
\begin{equation*}
	\psi(\v{r},\phi)=\sum\limits_{n=1}^{N_s}\sum\limits_{m=-N_d}^{N_d}
	X_{nm}\xi_{nm}(\v{r},\phi)
\end{equation*}
In this paper, the basis function is chosen as
\begin{align*}
	\xi_{nm}(\v{r},\phi) =& S_n(\v{r})e^{i m \phi}		\\ 
	S_n(\v{r},\phi) =& \begin{cases}
		1, & \v{r}\in S_n \\
		0, & \v{r}\notin S_n
	\end{cases}
\end{align*}
The latter sections, the following notations are used:
\begin{center} \begin{tabular}{lll}
	$S_n(\v{r})$	&	the pulse function		\\
	$S_n)$		&	the n-th triangle		\\
	$S(n)$		&	the area of the n-th triangle	\\
	$\sum\nolimits_{n,m}$	&
	$\sum\nolimits_{n=1}^{N_s}\sum\nolimits_{m=-N_d}^{N_d}$	\\
\end{tabular} \end{center}

%%%%%%%%%%%%%%%%%%%%%%%%%%%%%%%%%%%%%%%%%%%%%%%%%%%%%%%%%%%%%%%%%%%%%%%%%%%%%%%% 
\end{document}
