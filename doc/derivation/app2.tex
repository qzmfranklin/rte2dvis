\documentclass[main]{subfiles} 
\begin{document}
\section{Simplex Coordinates and Interpolation Polynomials}
\label{Simplex Coordinates and Interpolation Polynomials}
%%%%%%%%%%%%%%%%%%%%%%%%%%%%%%%%%%%%%%%%%%%%%%%%%%%%%%%%%%%%%%%%%%%%%%%%%%%%%%%%
%TODO
Given the coefficient $e_{ijk}$ at node $(i,j,k)$, the polynomial basis 
function $B_{ijk}(L_1,L_2,L_3)$ can be used to compute the value at any 
position in space.
\begin{equation*}
	E(L_1,L_2,L_3)=\sum\limits_{i=0}^M\sum\limits_{j=0}^{M-i}
	e_{ijk}B_{ijk}(L_1,L_2,L_3)
\end{equation*}

$B_{ijk}$ can be expressed in terms of the auxiliary polynomials $R_s$
\begin{align*}
	R_s(M,L)&=\frac{1}{s!}\prod\limits_{k=0}^{s-1}(M\,L-k)\,\,\,\,\,(s>0) \\
	R_0(M,L)&=1
\end{align*}
\begin{equation*}
	B_{ijk}=R_i(M,L_1)R_j(M,L_2)R_k(M,L_3)
\end{equation*}

The simplex coordinates are related to Euclidean coordinates by
\begin{equation*}
	\begin{pmatrix}
		x_1 & x_2 & x_3	\\
		y_1 & y_2 & y_3	\\
		1   & 1   & 1
	\end{pmatrix}
	\begin{pmatrix}
		L_1 \\ L_2 \\ L_3
	\end{pmatrix}
	=
	\begin{pmatrix}
		x \\ y \\ 1
	\end{pmatrix}
\end{equation*}

The Jacobian is $\pd{(x,y)}{(L_1,L_2)}=2\abs{A}$, where $A$ is the area of the
triangular domain being integrated. Also, a useful integral identity is
\begin{equation*}
	\int L_1^a L_2^b L_3^c dL_1 dL_2=\frac{a!b!c!}{(a+b+c+2)!}
\end{equation*}

%%%%%%%%%%%%%%%%%%%%%%%%%%%%%%%%%%%%%%%%%%%%%%%%%%%%%%%%%%%%%%%%%%%%%%%%%%%%%%%%
\end{document}
