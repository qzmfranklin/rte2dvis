\documentclass[main]{subfiles} 
\begin{document}
\section{Volume Integral Equation}
\label{sec:2d volume integral equation}
%%%%%%%%%%%%%%%%%%%%%%%%%%%%%%%%%%%%%%%%%%%%%%%%%%%%%%%%%%%%%%%%%%%%%%%%%%%%%%%%

In general, RTE can be written as:
\begin{equation*}
	\uv{s}\cdot\grad\psi(\v{r},\uv{s})
	+\mu_t(\v{r})\psi(\v{r},\uv{s})
	-\mu_s(\v{r})\int d\uv{s}^\prime
	f(\uv{s}\cdot\uv{s}^\p)\psi(\v{r},\uv{s}^\p)
	=q(\v{r},\uv{s})
\end{equation*}
with
\begin{center} \begin{tabular}{lll}
	\v{r}	&	$[L]^{+1}$	& position vector		\\
	\uv{s}	&	$[L]^{0}$	& unit direction vector		\\
	$\psi$	&	$[L]^{0}$	& specific intensity		\\
	$q$	&	$[L]^{-1}$	& source			\\
	$f$	&	$[L]^{0}$	& phase function		\\
	$\mu_s$	&	$[L]^{-1}$	& scattering cross-section	\\
	$\mu_a$	&	$[L]^{-1}$	& absorption cross-section	\\
	$\mu_t$	&	$[L]^{-1}$	& $\mu_a+\mu_s$, total cross-section	\\
\end{tabular} \end{center}

We can construct the following $I+K=V$ type of volume integral equation (VIE):
\begin{equation} \label{vie}
	\begin{split}
		\psi(\v{r},\uv{s}) +& \int d\v{r}^\p d\uv{s}^\p
		g_0(\v{r},\uv{s};\v{r}^\p,\uv{s}^\p)
		\mu_t(\v{r}^\p)\psi(\v{r}^\p,\uv{s}^\p)
		\\
		-& \int d\v{r}^\p d\uv{s}^\p g_0(\v{r},\uv{s};\v{r}^\p,\uv{s}^\p)
		\mu_s(\v{r}^\p) \int d\uv{s}^\pp
		f(\uv{s}^\p\cdot\uv{s}^\pp)\psi(\v{r}^\p,\uv{s}^\pp)
		\\
		=& \int d\v{r}^\p d\uv{s}^\p
		g_0(\v{r},\uv{s};\v{r}^\p,\uv{s}^\p)q(\v{r}^\p,\uv{s}^\p)
	\end{split}
\end{equation}
where $g$ is the free space Green's function:
\begin{equation} \label{g0}
	g_0(\v{r},\uv{s};\v{r}^\p,\uv{s}^\p) = \frac{1}{\abs{\v{r}-\v{r}^\p}} 
	\delta(\uv{s}-\uv{s}_{\v{r}-\v{r}^\p})
	\delta(\uv{s}^\p-\uv{s}_{\v{r}-\v{r}^\p})
\end{equation}

In 2D, \eqref{vie} and \eqref{g0} become
\begin{align} 
	\begin{split} 
		\label{2dvie}
		\psi(\v{r},\phi) +& \int d\v{r}^\p d\phi^\p
		g_0(\v{r},\phi;\v{r}^\p,\phi^\p)
		\mu_t(\v{r}^\p)\psi(\v{r}^\p,\phi^\p)
		\\
		-& \int d\v{r}^\p d\phi^\p g(\v{r},\phi;\v{r}^\p,\phi^\p)
		\mu_s(\v{r}^\p) \int d\phi^\pp
		f(\phi^\p-\phi^\pp)\psi(\v{r}^\p,\phi^\pp)
		\\
		=& \int d\v{r}^\p d\phi^\p
		g_0(\v{r},\phi;\v{r}^\p,\phi^\p)q(\v{r}^\p,\phi^\p)
	\end{split}
	\\
	\begin{split}
		\label{2dg0}
		g_0(\v{r},\phi;\v{r}^\p,\phi^\p) =& \frac{1}{\abs{\v{r}-\v{r}^\p}} 
		\delta(\phi-\phi_{\v{r}-\v{r}^\p})
		\delta(\phi^\p-\phi_{\v{r}-\v{r}^\p}) 
	\end{split}
\end{align}
%%%%%%%%%%%%%%%%%%%%%%%%%%%%%%%%%%%%%%%%%%%%%%%%%%%%%%%%%%%%%%%%%%%%%%%%%%%%%%%% 
\end{document}
