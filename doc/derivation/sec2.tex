\documentclass[main]{subfiles} 
\begin{document}
\section{Discretization}
\label{sec:discretization}
%%%%%%%%%%%%%%%%%%%%%%%%%%%%%%%%%%%%%%%%%%%%%%%%%%%%%%%%%%%%%%%%%%%%%%%%%%%%%%%%
Designate the volume of interest (scatter) as $\Omega$. Mesh $\Omega$. Denote 
the number of vertices, edges, and triangles as $N_v$, $N_e$, and $N_s$, 
respectively. Choose the proper spatial interpolation order $M>0$ and proper 
basis functions [Peterson]. The total number of nodes, i.e., the total number 
of degrees of freedom, is $N_n=N_v+(M-1)N_e+\frac{1}{2}(M-1)(M-2)N_s$.

Each spatial polynomial basis function associated with the $n$-th node, denoted
as $\xi_n(\v{r})$, generally speaking, consists of contributions from possibly
more than one triangle. Use the subscript $s$ to indicate a triangle associated 
with the $n$-th node. We have, in short, $\xi_n(\v{r})=\sum\nolimits_s
B_{ns}(\v{r})$. The $B_{ns}(\v{r})$ is the corresponding polynomial 
$B_{ijk}(L_1,L_2,L_3)$ [appendix] in terms of Euclidean coordinates [FIXME]. The
summation goes through all the triangles that share the $n$-th node.

Expand the specific intensity $\psi(\v{r},\phi)$ in terms of polynomial spatial 
basis functions $\xi_n(\v{r})$ ($n=1,...,N_n$) and harmonic angular basis 
functions $e^{i m \phi}$ ($m=-N_d,...,N_d$):
\begin{equation} \label{expansion of psi}
	\psi(\v{r},\phi)=\sum\limits_{nm}
	X_{nm}\xi_n(\v{r})e^{i\,m\phi}
\end{equation}

Plug \eqref{expansion of psi} into \eqref{2dvie}, multiply by
$\xi_n(\v{r},\phi)e^{-i\,m \phi}$, integrate $d\v{r}d\phi$ over the entire phase
space, we get a system of linear equations:
\begin{equation*}
	\sum\limits_{n^\p,m^\p}Z_{(nm)(n^\p m^\p)}X_{n^\p m^\p}=V_{nm}
\end{equation*}

The impedance matrix $Z=I+K$ consists of an identity term $I$ and an interaction
term $K$. The identity term $I$
\begin{equation*}
	\begin{split} 
		I_{(nm)(n^\p m^\p)} =& \int d\v{r}\,d\phi \,\xi_n(\v{r})
		e^{-i\,m\,\phi}\xi_{n^\p}(\v{r})e^{i\,m^\p\phi} \\
		=& 2\pi\,\delta_{mm^\p}\int d\v{r}\,\xi_n(\v{r})\xi_{n^\p}(\v{r})
		\\ =& 2\pi\,\delta_{mm^\p}\sum\limits_{s,s^\p} 
		\int d\v{r}\,B_{ns}(\v{r})B_{n^\p s^\p}(\v{r})
	\end{split}
\end{equation*}
The interaction term $K$
\begin{equation*}
	\begin{split}
		K_{(nm)(n^\p m^\p)} =& \int d\v{r}\,d\phi\,\xi_n(\v{r})
		e^{-i\,m\,\phi} \int d\v{r}^\p d\phi^\p
		g_0(\v{r},\phi;\v{r}^\p,\phi^\p) \mu_t(\v{r}^\p) 
		\xi_{n^\p}(\v{r}^\p) e^{i\,m^\p\phi^\p} \\
		-& \int d\v{r}\,d\phi\,\xi_n(\v{r}) e^{-i\,m\,\phi} 
		\int d\v{r}^\p d\phi^\p g_0(\v{r},\phi;\v{r}^\p,\phi^\p) 
		\mu_s(\v{r}^\p) \int d\phi^\pp f(\phi^\p-\phi^\pp)
		\xi_{n^\p}(\v{r^\p}) e^{i\,m^\p \phi^\pp} \\
		=& \int d\v{r}\,d\v{r}^\p \xi_n(\v{r}) \xi_{n^\p}(\v{r}^\p)
		\frac{e^{-i(m-m^\p)\phi_{\v{r}-\v{r}^\p}}}
		{\left\vert \v{r}-\v{r}^\p \right\vert}
		\left[ \mu_t(\v{r}^\p)-g^{|m^\p|}\mu_s(\v{r}^\p)\right] \\
		=& \sum\limits_{s,s^\p} 
		\int d\v{r}\,d\v{r}^\p B_{ns}(\v{r}) B_{n^\p s^\p}(\v{r}^\p)
		\frac{e^{-i(m-m^\p)\phi_{\v{r}-\v{r}^\p}}}
		{\left\vert \v{r}-\v{r}^\p \right\vert}
		\left[ \mu_t(\v{r}^\p)-g^{|m^\p|}\mu_s(\v{r}^\p)\right]
	\end{split}
\end{equation*}

The integral $\int d\v{r}\,B_{ns}(\v{r})B_{n^\p s^\p}(\v{r})$ can be evaluated
either analytically or numerically. 

The identity term, $I$, is a nearly diagonal sparse matrix.  

The integral in the interaction term can only be computed numerically [appendix].  

The \textit{r.h.s.} can be computed through a chain in the order of
\begin{equation*}
	q(\v{r},\phi) \to \psi^I(\v{r},\phi) \to V_{nm},
\end{equation*}
where $\psi^I(\v{r},\phi)=\int d\v{r}^\p d\phi^\p g(\v{r},\phi;\v{r}^\p,\phi^\p)
q(\v{r}^\p,\phi^\p)$ is the incident specific intensity. In general $\psi^I$ can
be computed given the source $q$. In this paper, however, we only deal with the
simpler case of planewave incidence. In case of planewave incidence, $q$ is 
infinite, but $\psi^I$ is finite. Normalize to $\psi^I(\v{r},\phi) = 
\delta(\phi-\phi^I)$. Then the \textit{r.h.s.} is just
\begin{equation*}
	V_{nm} = \int d\v{r}d\phi\,\xi_n(\v{r})e^{-i\,m\,\phi}\psi^I(\v{r},\phi) =
	e^{-i\,m\phi^I} \int d\v{r}\,\xi_n(\v{r})
\end{equation*}

%%%%%%%%%%%%%%%%%%%%%%%%%%%%%%%%%%%%%%%%%%%%%%%%%%%%%%%%%%%%%%%%%%%%%%%%%%%%%%%% 
\end{document}
