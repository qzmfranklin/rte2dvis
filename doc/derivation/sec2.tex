\documentclass[main]{subfiles} 
\begin{document}
\section{Discretization}
\label{sec:discretization}
%%%%%%%%%%%%%%%%%%%%%%%%%%%%%%%%%%%%%%%%%%%%%%%%%%%%%%%%%%%%%%%%%%%%%%%%%%%%%%%%
Designate the volume of interest (scatter) as $\Omega$. Mesh $\Omega$. Denote 
the number of vertices, edges, and triangles as $N_v$, $N_e$, and $N_s$, 
respectively. Choose the proper spatial interpolation order $M>0$ and proper 
basis functions [Peterson]. The total number of nodes, i.e., the total number 
of degrees of freedom, is $N_n=N_v+(M-1)N_e+\frac{1}{2}(M-1)(M-2)N_s$.

Each spatial polynomial basis function associated with the $n$-th node, denoted
as $\xi_n(\v{r})$, generally speaking, consists of contributions from possibly
more than one triangle. Use the subscript $s$ to indicate a triangle associated 
with the $n$-th node. We have, in short, $\xi_n(\v{r})=\sum\nolimits_s
B_{ns}(\v{r})$. The $B_{ns}(\v{r})$ is the corresponding polynomial 
$B_{ijk}(L_1,L_2,L_3)$ [appendix] in terms of Euclidean coordinates [FIXME]. The
summation goes through all the triangles that share the $n$-th node.

Expand the specific intensity $\psi(\v{r},\phi)$ in terms of polynomial spatial 
basis functions $\xi_n(\v{r})$ ($n=1,...,N_n$) and harmonic angular basis 
functions $e^{i m \phi}$ ($m=-N_d,...,N_d$):
\begin{equation} \label{expansion of psi}
	\psi(\v{r},\phi)=\sum\limits_{nm}
	X_{nm}\xi_n(\v{r})e^{i\,m\phi}
\end{equation}

Plug \eqref{expansion of psi} into \eqref{vie2d}, multiply by
$\xi_n(\v{r},\phi)e^{-i\,m \phi}$, integrate $d\v{r}d\phi$ over the entire phase
space, we get a system of linear equations:
\begin{equation*}
	\sum\limits_{n^\p,m^\p}Z_{(nm)(n^\p m^\p)}X_{n^\p m^\p}=V_{nm}
\end{equation*}
with
\begin{align*}
	\begin{split} 
		Z_{(nm)(n^\p m^\p)} =& I_{(nm)(n^\p m^\p)}+K_{(nm)(n^\p m^\p)}
	\end{split}
	\\
	\begin{split} 
		I_{(nm)(n^\p m^\p)} =& \int d\v{r}d\phi\xi_n(\v{r})e^{-i\,m\phi}
		\int d\v{r}^\p d\phi^{\p}\xi_{n^\p}(\v{r}^\p)e^{i\,m^\p\phi^\p} 
	\end{split}
	%\\
	%\begin{split} 
		%K_{(nm)(n^\p m^\p)} =& \int d\v{r}d\phi\xi_{nm}^{*}(\v{r},\phi)
		%\int d\v{r}^\p d\phi^\p g(\v{r},\phi;\v{r}^\p,\phi^\p) \\
		%\times& \mu_t(\v{r}^\p)\xi_{n^\p m^\p}(\v{r}^\p,\phi^\p) -
		%\int d\v{r}d\phi\xi_{nm}^{*}(\v{r},\phi) \\ \times&
		%\int d\v{r}^\p d\phi^\p g(\v{r},\phi;\v{r}^\p,\phi^\p)
		%\mu_s(\v{r}^\p) \\ \times& \int d\phi^\pp
		%f(\phi^\p-\phi^\pp)\xi_{n^\p m^\p}(\v{r}^\p,\phi^\pp)
	%\end{split}
	%\\
	%V_{nm} =& \int d\v{r}d\phi\xi_{nm}^{*}(\v{r},\phi)\psi^I(\v{r},\phi)
	%\\
	%\psi^I(\v{r},\phi) \equiv& \int d\v{r}^\p d\phi^\p
	%g(\v{r},\phi;\v{r}^\p,\phi^\p)q(\v{r}^\p,\phi^\p)
\end{align*}

Note that $(n,m)$ represents the filed point, whereas $(n^\p,m^\p)$ represents 
the source point.
%%%%%%%%%%%%%%%%%%%%%%%%%%%%%%%%%%%%%%%%%%%%%%%%%%%%%%%%%%%%%%%%%%%%%%%%%%%%%%%% 
\end{document}
