\documentclass[main]{subfiles} 
\begin{document}
\section{Section4}
\label{sec:Section4}
%%%%%%%%%%%%%%%%%%%%%%%%%%%%%%%%%%%%%%%%%%%%%%%%%%%%%%%%%%%%%%%%%%%%%%%%%%%%%%%%
Assuming planewave incidence, numerically compute the right hand side 
(\textit{r.h.s.}), i.e., the input vector:
\begin{equation*} 
	V_{nm} = \sum\limits_{n^\p}^{N_s} D(n,n^\p,m)
\end{equation*}
where $D$ is a rank-3 tensor:
\begin{equation*} \begin{split} 
	D(n,n^\p,m) =& \sum\limits_{j_n=1}^{M_n}
	\sum\limits_{j_{n^\p}=1}^{M_{n^\p}}
	w_{j_n}w_{j_{j^\p}}\mu_s(\v{r}_{j_{n^\p}}) 
	\frac{e^{-i m\phi_{\v{r}_{j_n}-\v{r}_{j_{n^\p}}}}}
	{\abs{\v{r}_{j_n}-\v{r}_{j_{n^\p}}}} \\ \times&
	f(\phi_{\v{r}_{j_n}-\v{r}_{j_{n^\p}}})
	e^{-\tau(\v{r}_{j_{n^\p}},-\uv{s}^I)}
\end{split} \end{equation*}

$\tau(\v{r},-\uv{s}^I)\equiv\int_{C_{\v{r},-\uv{s}^I}}\mu_t dl$, where the
integral path $C_{\v{r},-\uv{s}^I}$ is the ray that starts from $\v{r}$ and in
the direction pointed by $-\uv{s}^I$. For later use,
$\tau(\v{r}^\p\to\v{r}) \equiv \int_{C_{\v{r}^\p\to\v{r}}}$, where
$C_{\v{r}^\p\to\v{r}}$ is the line segment from $\v{r}^\p$ to $\v{r}$.

In principle, computing $\tau$ requires ray tracing. Done with a tree-structure
[TODO].
%%%%%%%%%%%%%%%%%%%%%%%%%%%%%%%%%%%%%%%%%%%%%%%%%%%%%%%%%%%%%%%%%%%%%%%%%%%%%%%%
\end{document}
